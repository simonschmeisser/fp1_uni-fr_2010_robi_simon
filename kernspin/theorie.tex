\section{Theoretische Grundlagen}

\subsection{Kernspin}

Neben den Elektronen, besitzen auch die Protonen und Neutronen einen Spin, einen intrinistischen, diskreten Drehimpuls $\vec I$. Dieser ist gegeben durch:

$$ |\vec I| = \hbar \sqrt{I(I+1)} $$

In z-Richtung (in unserem Fall auch die Richtung des Feldes des Permanentmagnetes) kann der Spin auch nur ganzzahlige Vielfache von $\hbar$ annehmen, nämlich

$$I_z = \hbar\cdot m_I$$

wobei $m_I$ die magnetische Spinquantenzahl ist und nur ganzzahlige Werte zwischen $-I$ und $I$ annehmen kann. Es gibt also $2I+1$ verschiedene Spinzustände. Bei Fermionen ist $I=1/2$ und daher $m_I = \pm\frac{1}{2}$, es gibt also zwei verschiedene Spineinstellungsmöglichkeiten für ein Energieniveau.

\subsubsection{Magnetisches Moment}

Da der Spin als Eigendrehimpuls eines Teilchens aufgefaßt werden kann, induziert ein Teilchen mit Spin und einer Ladung ein magnetisches Dipolmoment, welches durch das gyromagnetische Verhältnis $\gamma$ direkt mit dem Spin zusammenhängt. (Wir erläutern hier nur die spezifischen Formeln für den Kernspin der Protons).

\begin{equation} \vec \mu_I = \gamma\cdot\vec I = \frac{g_K\cdot\mu_K}{\hbar}\vec I \end{equation}

$\mu_K$ ist hier das Kernmagneton und ist eine Konstante:

$$\mu_K = \frac{e\cdot \hbar}{2\cdot m_p} \approx 3.15\e{-8} eV/T$$

und $m_p$ ist die Masse des Protons. In z-Richtung gilt somit für das magnetische Moment:

\begin{equation} \mu_z = g_K \cdot \mu_K \cdot m_I \label{muz} \end{equation}

\subsubsection{Spin im Magnetfeld}

Die potentielle Energie eines magnetischen Moments in einem Magnetfeld $\vec B$ ist gegeben durch:

$$ E_p = -\vec\mu\cdot\vec B $$

Legt man das $\vec B$-Feld o.B.d.A in z-Richtung, so erhält man folgenden Zusammenhang:

$$ E_p = -\mu_z\cdot B_z \stackrel{(\ref{muz})}{=} -g_K \cdot \mu_K \cdot m_I\cdot B_z $$

Die Energie eines spinbesetzten Teilchens in einem $\vec B$-Feld hängt also von der Stärke des $\vec B$-Felds ab und von dessen magnetischen Spinquantenzahl, also seines Spin-Zustandes. Spin-up-Teilchen ($m_I = +1/2$) haben also eine andere Energie als Spin-down-Teilchen ($m_I = -1/2$). Dies bezeichnet man als Zeeman-Aufspaltung der Spinzustände. Der Energieunterschied zwischen eines Spin-up- und eines Spin-down-Teilchens ergibt sich also durch:

\begin{equation} \Delta E = E_p(\uparrow) - E_p(\downarrow) =  g_K \cdot \mu_K \cdot B_z \label{de} \end{equation}

Dies ist die Energie die man braucht, bzw. erhält, wenn ein Spin im Magnetfeld umklappt.

\subsection{Resonanz}

Die Energie aus (\ref{de}) kann man natürlich auch mit elektromagnetischer Strahlung erreichen. Hat die Strahlung genau die richtige Frequenz, die sogenannte Resonanzfrequenz $\nu_r$, so kann sie Spin-Übergänge induzieren. In diesem Fall gilt:

$$ \Delta E = E_\nu \ \Leftrightarrow \ g_K \cdot \mu_K \cdot B_z = h\cdot \nu_r $$

Die Resonanzfrequenz ist also gegeben durch:

$$ \nu_r = \frac{g_K \cdot \mu_K \cdot B_z}{h} $$

Man spricht in diesem Fall von Kernspinresonanz.



%gyromagn. Verhältnis
%Energieniveaus, Boltzmann-Verteilung

\subsection{Kernspinresonanz}

%klassisch
%qm
%Relaxation
%Linienbreite
%Sättigung
%Empfindlichkeit
%magnetische Dipol-Dipol-ww
%Kernspinresonanzspektrometer

\subsection{Hall-Effekt und Hall-Sonde}

\subsection{Das Lock-In Verfahren}

\subsection{Synchrondetektor}