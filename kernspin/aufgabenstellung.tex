\section{Aufgabenstellung}

\begin{enumerate}
\item Bestimmen Sie das Magnetfeld des Permanentmagneten mit einer Hallsonde und 
    untersuchen Sie die Homogenität des Feldes. Stellen Sie das Ergebnis in zwei 
    Hauptschnitten graphisch dar. Wählen Sie sinnvolle Maßstäbe, um auch kleine 
    Abweichungen vom Nennwert des Feldes noch erkennen zu können. 
\item Bestimmen Sie das gyromagnetische Verhältnis des Protons in einer Glykolprobe. 
    Untersuchen Sie die Homogenität des Magnetfeldes mit der Glykolprobe und 
    vergleichen Sie das Ergebnis mit den unter 1. erzielten Werten. 
\item Untersuchen Sie die Protonenresonanz in einer Wasserstoffprobe. Versuchen Sie 
    auch hier eine Bestimmung der Resonanzfrequenz und vergleichen Sie sie mit 
    dem Ergebnis aus 2. 
\item Bestimmen Sie das kernmagnetische Moment des $^{19}F$-Kerns in einer Teflonprobe. 
\item Zeichnen Sie die Protonen-Resonanz in der Glykolprobe auf, indem Sie das 
    Magnetfeld langsam über die Resonanz fahren und mit Hilfe des Lock-in 
    Verfahrens (Synchrondetektor) ein differenziertes Resonanzsignal erzeugen.
\end{enumerate}