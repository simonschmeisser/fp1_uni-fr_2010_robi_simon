\clearpage
\section{Zusammenfassung}

Im ersten Teil des Versuchs haben wir das Magnetfeld des Permanentmagneten mit einer Hallsonde vermessen, um herauszufinden, an welchen Stellen das Magnetfeld homogen ist, und wie stark das Feld an dieser Stelle ist. Die Ermittlung der Homogenität haben wir bei den weiteren Messungen benutzt, die Magnetfeldstärke beträgt:

$$ \boxed{B = (434.1 \pm 4.5)\ mT} $$

Dann haben wir versucht die Resonanzfrequenz der 3 vorliegenden Proben zu messen, indem wir die Frequenz so verändert haben, dass die Absorptionslinien äquidistant waren. Diese Frequenz wurde dann gemessen und daraus der Landé-Faktor $g$ und das gyromagnetische Verhältnis $\gamma$ berechnet:


\begin{center}
\begin{tabular}{| l | c | c | c | c |} \hline
Probe & $\nu_r$ / MHz & $g$ & $\sigma(g)$ & $\gamma\ / MHz\cdot T^{-1}$\\ \hline
Glykol & $18.184 \pm 0.050$ & $5.495 \pm 0.059$ & 2 & $263.2 \pm 2.8$ \\
Wasserstoff &$18.180 \pm 0.002$ & $ 5.494 \pm 0.057 $ & 2 & $263.2 \pm 2.7$ \\
Teflon & $17.114 \pm 0.002$ &  $5.172 \pm 0.054$ & 2 & $247.7 \pm 2.6$\\ \hline
\end{tabular}
\end{center}

Für die Teflonprobe haben wir dann noch das kernmagnetische Moment $\mu_K$ berechnet, indem wir den Wert für $g$ als gegeben annahmen. Wir erhielten den Wert:

$$\boxed{\mu_K = (4.968 \pm 0.052)\cdot 10^{-27} J/T} $$

Der theoretische Wert liegt in der 2.Standard-Abweichung von diesem Wert.

Mithilfe der Wasserstoffprobe haben wir eine zweite Messung der Homogenität des Magnetfeldes durchgeführt und einen leichte lineare Zunahme der Magnetfeldstärke in x-Richtung bzw. eine lineare Abnahme in y-Richtung auf dem vorhin ermittelten Plateau feststellen können.

Im letzten Teil des Versuchs haben wir anhand des Lock-In-Verfahrens durch einen linearen Fit noch ein Mal die Resonanzfrequenz der Wasserstoffprobe gemessen und erhielten:

$$\boxed{\nu_r = (18.1893 \pm 0.0019)\ MHz}$$

Dieser Wert liegt sehr dicht an der vorher ermittelten Resonanzfrequenz (0.051\% Abweichung). Weiterhin konnten wir mit dieser Resonanzfrequenz auch den Landé-Faktor berechen und erhielten
$$ g = 5.497 \pm 0.057 $$ 

welcher in der 2. Standardabweichung des Literaturwerts $g_{lit,H} = 5.5856$ liegt.