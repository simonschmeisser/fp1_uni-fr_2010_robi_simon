\section{Theoretische Grundlagen}

\subsection{Polarisationszustände des Lichts}

Aus den Maxwellgleichungen ergeben sich als allgemeine Lösungen die ebenen Wellen, gegeben durch
$$\vec E = \vec{E_0}\cdot \cos(\omega t - kz)$$
wobei o.B.d.A k in z-Richtung liegt. Der Vektor $\vec{E_0} \perp \vec k$ beschreibt somit wie das E-Feld orientiert ist. Bei einer Lichtwelle gilt genau das gleiche noch einmal für das B-Feld, darauf gehen wir aber jetzt nicht näher ein, da es komplett analog ist.

Für $\vec{E}$ gibt es 3 mögliche Polarisationszustände:

\begin{itemize}

\item Lineare Polarisation: Das E-Feld zeigt immer in die gleiche Richtung, d.h. die $x$- und die $y$-Komponenten schwingen in Phase:
$ \vec{E} = \begin{pmatrix} E_{0x} \\ E_{0y} \\ 0 \end{pmatrix}\cdot\cos(\omega t-kz) $

\item Zirkulare Polarisation: Das E-Feld dreht sich um seine Propagationsachse, d.h. dass die Beträge der $x$- un $y$-Komponenten gleich sind, ihre Phase jedoch um 90$^\circ$ verschoben ist:
$\vec{E}= E_0\cdot \begin{pmatrix} \cos(\omega t - kz) \\ \sin(\omega t - kz) \\ 0 \end{pmatrix}$

\item Elliptische Polarisation: In allen anderen Fällen ist das Licht elliptisch polarisiert.
\end{itemize}


\subsection{Doppelbrechung}

In einem anisotropen Medium sind Ausbreitungsrichtung der Welle und Energieflussrichtung verschieden. Der Brechungsindex ist abhängig von der Raumrichtung und somit werden verschieden polarisierte Wellen anders gebrochen. Einfallendes Licht wird in zwei Teilbündel gespalten, wovon ein Teil dem Snelliusschen Brechungsindex beim Eintreten in das Medium gehorcht (ordentlicher Strahl), der andere Teil nicht (außerordentlicher Strahl). Beide Strahlen sind senkrecht zueinander polarisiert. Die ordentliche Welle ist senkrecht und die außerordentliche Welle ist parallel zu optischen Achse des Kristalls polarisiert.
















\clearpage