\section{Theoretische Grundlagen}

\subsection{Polarisationszustände des Lichts}

Aus den Maxwellgleichungen ergeben sich als allgemeine Lösungen die ebenen Wellen, gegeben durch
$$\vec E = \vec{E_0}\cdot \cos(\omega t - kz)$$
wobei o.B.d.A k in z-Richtung liegt. Der Vektor $\vec{E_0} \perp \vec k$ beschreibt somit wie das E-Feld orientiert ist. Bei einer Lichtwelle gilt genau das gleiche noch einmal für das B-Feld, darauf gehen wir aber jetzt nicht näher ein, da es komplett analog ist.

Für $\vec{E}$ gibt es 3 mögliche Polarisationszustände:

\begin{itemize}

\item Lineare Polarisation: Das E-Feld zeigt immer in die gleiche Richtung, d.h. die $x$- und die $y$-Komponenten schwingen in Phase:
$ \vec{E} = \begin{pmatrix} E_{0x} \\ E_{0y} \\ 0 \end{pmatrix}\cdot\cos(\omega t-kz) $

\item Zirkulare Polarisation: Das E-Feld dreht sich um seine Propagationsachse, d.h. dass die Beträge der $x$- un $y$-Komponenten gleich sind, ihre Phase jedoch um 90$^\circ$ verschoben ist:
$\vec{E}= E_0\cdot \begin{pmatrix} \cos(\omega t - kz) \\ \sin(\omega t - kz) \\ 0 \end{pmatrix}$

\item Elliptische Polarisation: In allen anderen Fällen ist das Licht elliptisch polarisiert.
\end{itemize}


\subsection{Doppelbrechung}

\begin{figure}[H]
	\centering \includegraphics[width = 0.8\textwidth]{Bilder/Doppelbrechung.jpg}
	\caption{Doppelbrechung}
\end{figure}

In einem anisotropen Medium sind Ausbreitungsrichtung der Welle und Energieflussrichtung verschieden. Der Brechungsindex ist abhängig von der Raumrichtung und somit werden verschieden polarisierte Wellen anders gebrochen. Einfallendes Licht wird in zwei Teilbündel gespalten, wovon ein Teil dem Snelliusschen Brechungsindex beim Eintreten in das Medium gehorcht (ordentlicher Strahl), der andere Teil nicht (außerordentlicher Strahl). Beide Strahlen sind senkrecht zueinander polarisiert. Die ordentliche Welle ist senkrecht und die außerordentliche Welle ist parallel zu optischen Achse des Kristalls polarisiert.


\subsection{Piezoeffekt}

In bestimmten Materialien, piezoelektrisch genannt, entstehen durch Verformung mikroskopische Dipole in diesen, die eine Spannung induzieren. Da sehr viele dieser Dipole entstehen, ist die Spannung sogar messbar. Legt man an piezoelektrische Materialien eine Spannung an, so verformen sich diese.


\subsection{Pockelseffekt}

Die dielektrische Konstante ist definiert durch 

\begin{equation} \epsilon = \frac{\partial D}{\partial E}  \end{equation}
\begin{equation} \text{mit \ } D = aE + bE^2 + cE^3 + \dots \end{equation}

$\epsilon$ ist also eigentlich keine Konstante und $D$ hängt nicht linear von $E$ ab. $a$, $b$ und $c$ sind hier einfach Konstanten. Für $\epsilon$ folgt also:
\begin{equation} \epsilon = a + 2bE + 3cE^2 + \dots \end{equation}

Der Term $3cE^2$ ist verantwortlich für den Kerr-Effekt und für unseren Versuch unwichtig. Alle Terme höherer Ordnung sind vernachlässigbar klein. Der Term $2bE$ ist verantwortlich für den Pockelseffekt. Die Dielektrizitätskonstante ist also, wie man sehen kann, (etwa) linear abhängig vom E-Feld, welches an das betrachtete Medium angelegt wird. Somit ist auch der Brechungsindex abhängig vom E-Feld, dies nennt man den Pockelseffekt. Der lineare Term $2bE$ ist der Term der beim Pockelseffekt wichtig ist. Dieser Effekt existiert nur für Kristalle ohne Symmetriezentrum. Im Vorfaktor $2b$ steckt der Piezoeffekt drin. 


\subsection{Faradayeffekt}

Legt man ein durchsichtiges isotropes Medium in ein longitudinales Magnetfeld, so wird das Medium optisch aktiv. Rechtszirkulare Lichtwellen und linkszirkulare Lichtwellen propagieren mit verschiedenen Geschwindigkeiten. Schickt man also eine linear polarisierte Wellen durch das Medium (linear polarisierte Wellen bestehen aus gegenseitig zirkular polarisierten Wellen), so kommt diese wegen der Phasenverschiebung der beiden Teilwellen in einem Winkel $\alpha$ zu dem Winkel, mit dem es eingefallen ist, heraus. Der Winkel $\alpha$ ist gegeben durch:

\begin{equation} \alpha = V\cdot l \cdot H \end{equation}

$H$ ist hier die magnetische Feldstärke des longitudinalen Feldes und $l$ die Länge des Mediums. $V$ wird als Verdet-Konstante bezeichnet und hängt von der Wellenlänge und der Dispersion ab.

\subsection{Magnetfeld einer Spule}





























\clearpage