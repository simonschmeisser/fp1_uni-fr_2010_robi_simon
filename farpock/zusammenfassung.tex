\section{Zusammenfassung}

\paragraph{Pockelseffekt} Wir konnten die Auswirkungen des Pockelseffektes relativ genau bestimmen, insbesondere mit der Sägezahnmethode erhielten wir gute Ergebnisse:

\begin{center}
\fbox{\vbox{\hsize=10cm\noindent
$r_{41_r} = (22.473 \pm 0.154) pm/V$  (Robi - Sinus)\\
$r_{41_s} = (21.969 \pm 0.173) pm/V$  (Simon - Sinus)\\
$r_{41} = (23.049 \pm 0.003) pm/V$  (Sägezahn)\\
}}
\end{center}

Zum Vergleich nochmals der Literaturwert

\begin{center}
\fbox{\vbox{\hsize=10cm\noindent
$r_{41_l} = 23.4 pm/V$\\
}}
\end{center}

\paragraph{Faradayeffekt} Wir konnten die Polarisationsdifferenz des Halbschattenpolarimeters bestimmen mit:

\begin{center}
\fbox{\vbox{\hsize=10cm\noindent
$ 2\epsilon = 13.60^\circ $ (Simon) \\
$ 2\epsilon = 12.45^\circ $ (Robi) \\
}}
\end{center}

Die Angabe des Herstellers zur Verdetkonstante des Flintstabes konnten wir bestätigen und erhielten

\begin{center}
\fbox{\vbox{\hsize=10cm\noindent
$ V = (0.0478 \pm 0.0024) \frac{min}{Oe \cdot cm} $ (Simon) \\
$ V = (0.0482 \pm 0.00242) \frac{min}{Oe \cdot cm} $ (Robi) \\
}}
\end{center}
