\section{Zusammenfassung}

Wir haben uns in diesem Versuch mit den Prinzipien der Rastertunnelmikroskopie vertraut gemacht. Wir haben die Spitzen für Messung selber gemacht und in das Mikroskop eingesetzt. Damit haben wir versucht die Topographie von drei verschiedene Proben zu messen und graphisch darzustellen.

Das erste Material war eine Goldprobe. In der Auswertung kann man Graphiken von 3 verschiedenen Bildgrößen einsehen, nämlich 500 nm, 150 nm und atomar (3.5 nm). Beim Gold war es jedoch nicht möglich eine Gitterkonstante zu messen, da wir, wie erwartet, keine atomare Auflösung erhielten.

Als zweite Probe nahmen wir Graphit, welches wir mit einer viel besseren Auflösung als das Gold messen konnten. Wir erhielten sehr saubere Bilder auf atomaren Niveau (siehe Auswertung), aus denen wir Gitterkonstanten messen konnten. Wir erhielten die Werte

\begin{align*}	
	&g^{F3} = (2.391 \pm 0.063) \mathring{A}\\
	&g^{F4}_1 = (2.391 \pm 0.063) \mathring{A}\\
	&g^{F4}_2 = (2.333 \pm 0.063) \mathring{A}\\
	&g^{F5}_1 = (2.341 \pm 0.063) \mathring{A}\\
	&g^{F5}_2 = (2.324 \pm 0.071) \mathring{A}
\end{align*}

die alle in der 2. bzw. 3. Standardabweichung vom theoretischen Wert $g_{Theo} = 2.456\ \mathring A$ liegen.

Die dritte Probe war Molybdändisulfid, welche wir in den Größen 50 nm und 5 nm in der Auswertung dargestellt haben. Aus den beiden Bildern in atomarer Auflösung konnten wir wiederum die Gitterkonstante ermitteln und erhielten:

\begin{align*}
	&g^{F7} = (2.744 \pm 0.100) \mathring{A}\\ 
	&g^{F8} = (2.700 \pm 0.056) \mathring{A}
\end{align*}

Der theoretische Wert beträgt $g_{Theo} = 3.16\ \mathring A$. Somit liegt der obere Wert in der 4., der untere in der 9. Standardabweichung.