\section{Durchführung}


Wir haben den PC hochgefahren und dann das Messprogramm \emph{Nanosurf easyscan 2} gestartet. Dann haben wir eine Spitze anhand einer Kneifzange aus dem $PtIr$-Draht hergestellt und mit einer Pinzette in das Rastertunnelmikroskop eingeführt. Unsere erste Messung haben wir mit Graphit durchgeführt. Wir haben bei jeder Messung die Spitze grob an die Probe herangefahren. 

In diesem Fall haben wir für die feinere Annäherung wir folgende Schritte durchgeführt:



\begin{table}[H]
\caption{Graphit\_1.png}
\centering \begin{tabular}[H]{l c c l}
Funktion & Approach Speed & Tip Voltage & Resultat/Bemerkung\\ \hline
Approach & 52\% & 0.15 V & grün-rot blinkend\\
Withdraw & & & orange\\
Approach & 29\% & 0.15 V & grün\\
 & 21\% & 90 mV & grün\\
 & 20\% & 50 mV & grün\\
\end{tabular}
\end{table}


Wir haben dann ein Bild aufgenommen mit folgenden Parametern:

\parameter{1.5}{0.4}{256}

\bild{Graphit_1}{Graphit\_1.png}

Nach diesem ziemlich verfehlten Bild haben wir eine neue Spitze hergestellt und eingesetzt. Wir haben eine erneute Messung des Graphits durchgeführt: 

\begin{table}[H]
\caption{Graphit\_2.png}
\centering \begin{tabular}[H]{l c c l}
Funktion & Approach Speed & Tip Voltage & Resultat/Bemerkung\\ \hline
Approach & 50\% & 0.15 V & orange\\
 & 40\% & 0.15 V & rot\\
Retract & & & orange\\
Approach & 25\% & 0.15 V & orange\\
 & 20\% & 0.15 V & orange - abgebrochen\\
 & 27\% & 0.15 V & orange\\
 & 24\% & 0.15 V & orange\\
 & 23\% & 90 mV & orange - abgebrochen\\
 & 25\% & 90 mV & grün\\
 & 20\% & 50 mV & grün\\
 & 20\% & 25 mV & grün\\
\end{tabular}
\end{table}

Wir haben immer dann abgebrochen, wenn das Heranfahren der Nadel viel zu lange gedauert hat, also deutlich einige Minuten überschritten hat. Wir haben das Bild \emph{Graphit\_2.png} mit folgenden Parametern aufgenommen: 

\parameter{1.5}{0.4}{256}

In der Auswertung kann man alle relevanten Bilder einsehen. Wir beschreiben hier nur wie wir vorgegangen sind, um die Bilder aufzunehmen. Die Namen der Bilder entsprechen den angegebenen Namen über den Tabellen.

Wir haben in das Bild \emph{Graphit\_2.png} hineingezoomt und nach dem Feineinstellen nochmal eine genauere Messung durchgeführt. (x-Position = -1.3 nm , y-Position = -42 nm). Wir mussten hier nochmal die Nadel neu heranführen, da die Kontrolllampe orange war. 

\begin{table}[H]
\caption{Graphitmessung}
\centering \begin{tabular}[H]{l c c l}
Funktion & Approach Speed & Tip Voltage & Resultat/Bemerkung\\ \hline 
Approach & 46\% & 25 mV & rot\\
Withdraw & & & orange\\
Approach & 35\% & 25 mV & grün\\
\end{tabular}
\end{table}

\parameter{1.5}{0.8}{512}

An dieser Stelle haben wir wieder eine neue Spitze gemacht, da dieses Bild wieder sehr unklar war, was vielleicht daran liegen kann, dass wir beim ersten Approach die Probe mit der Spitze berührt haben (rote Kontrollleuchte), und somit vielleicht die Spitze deformiert haben. Wir haben es nochmal versucht:

\begin{table}[H]
\caption{Graphitmessung}
\centering \begin{tabular}[H]{l c c l} 
Funktion & Approach Speed & Tip Voltage & Resultat/Bemerkung\\ \hline
Approach & 46\% & 25 mV & rot\\
Withdraw & & & orange\\
Approach & 35\% & 25 mV & grün\\
\end{tabular}
\end{table}

\parameter{1.5}{0.6}{512}

und nochmal mit neuer Spitze:

\begin{table}[H]
\caption{Graphitmessung}
\centering \begin{tabular}[H]{l c c l} 
Funktion & Approach Speed & Tip Voltage & Resultat/Bemerkung\\ \hline
Approach & 40\% & 0.1 V & rot\\
Withdraw & & & orange\\
Approach & 30\% & 0.1 V & grün\\
 & 20\% & 25 mV & grün\\
\end{tabular}
\end{table}

\parameter{5.58}{0.6}{256}

Da alle diese Bilder ziemlich nutzlos waren sind wir an dieser Stelle aus frustrationstechnischen Gründen auf eine andere Probe umgestiegen, nämlich das 160-nm-Gitter (\emph{Nano-Grid}) und haben wiederum eine neue Spitze angefertigt. Das Heranführen der Nadel war jedoch erfolglos, da das Rastertunnelmikroskop keinen Tunnelstrom registrieren konnte. Wir haben somit unsere Messungen mit Molybdändisulfid fortgeführt.

\begin{table}[H]
\caption{MoS2\_1.png, MoS2\_2.png, MoS2\_3.png}
\centering \begin{tabular}[H]{l c c l} 
Funktion & Approach Speed & Tip Voltage & Resultat/Bemerkung\\ \hline
Approach & 50\% & 0.15 V & rot\\
Withdraw & & & orange\\
Approach & 30\% & 0.15 V & rot\\
Withdraw & & & orange\\
Approach & 20\% & 25 mV & grün\\
 & 20\% & 0.90 V & grün\\
\end{tabular}
\end{table}

Wir haben 3 Bilder mit jeweils verschiedenen Zooms bei dieser Einstellung gemacht. Die Bildparameter lauten:

MoS2\_1.png \parameter{351.4}{0.4}{128}
MoS2\_2.png, \ \ \ Zoom: x-Pos = -17 nm , y-Pos = -0.17 nm \parameter{37.06}{0.4}{128}
MoS2\_3.png, \ \ \ Zoom: x-Pos = -16 nm , y-Pos = -0.17 nm \parameter{12.31}{0.4}{128}

Nach einem erneuten vergeblichen Versuch, das Nano-Grid zu messen, haben wir eine weitere Messung des $MoS_2$ gemacht. Der Approach lief folgendermaßen ab:

\begin{table}[H]
\caption{Molybdändisulfidmessung}
\centering \begin{tabular}[H]{l c c l} 
Funktion & Approach Speed & Tip Voltage & Resultat/Bemerkung\\ \hline
Approach & 40\% & 0.15 V & rot\\
Withdraw & & & orange\\
Approach & 25\% & 0.1 V & orange-abgebrochen\\
 & 30\% & 0.1 V & orange-abgebrochen\\
 & 40\% & 0.1 V & grün\\
\end{tabular}
\end{table}

Nachdem die Messung begonnen hatte, wurde das gesamte Bild auf einmal weiß und die Kontrollleuchte zeigte rot an. Wir haben einen neuen Approach versucht:

\begin{table}[H]
\caption{MoS2\_4.png}
\centering \begin{tabular}[H]{l c c l}
Funktion & Approach Speed & Tip Voltage & Resultat/Bemerkung\\ \hline
Approach & 30\% & 0.1 V & orange - abgebrochen\\
 & 35\% & 0.1 V & grün
\end{tabular}
\end{table}

\parameter{10.2}{0.2}{128}

Mit einer neuen Spitze haben wir wiederum die Graphit-Probe analysiert:

\begin{table}[H]
\caption{Graphit\_3.png}
\centering \begin{tabular}[H]{l c c l}
Funktion & Approach Speed & Tip Voltage & Resultat/Bemerkung\\ \hline
Approach & 51\% & 0.19 V & rot\\
Withdraw & & & orange\\
Approach & 35\% & 0.15 V & rot\\
Withdraw & & & orange\\
Approach& 20\% & 0.1 V & rot\\
Withdraw & & & orange\\
Approach & 39\% & 0.1 V & grün\\
\end{tabular}
\end{table}

\parameter{1.96}{0.5}{256}

Nach dem weiteren Anfertigen einer neuen Spitze, haben wir nochmal die Graphit-Probe benutzt:

\begin{table}[H]
\caption{Graphit\_4.png}
\centering \begin{tabular}[H]{l c c l}
Funktion & Approach Speed & Tip Voltage & Resultat/Bemerkung\\ \hline
Approach & 36\% & 0.15 V & grün\\
 & 30\% & 90 mV & grün\\
 & 25\% & 60 mV & grün\\
 & 20\% & 45 mV & grün
\end{tabular}
\end{table}

\parameter{2}{0.5}{128}

Wir haben wiederum eine neue Spitze angefertigt und mit der Messung des Goldes begonnen. 

\begin{table}[H]
\caption{Gold\_0, \_1, \_2, \_3.png}
\centering \begin{tabular}[H]{l c c l}
Funktion & Approach Speed & Tip Voltage & Resultat/Bemerkung\\ \hline
Approach & 31\% & 0.1 V & grün\\
 & 21\% & 45 mV & grün\\
\end{tabular}
\end{table}

Gold\_0.png \parameter{3.5}{0.5}{128}
Gold\_1.png \parameter{3.5}{0.5}{128}
Gold\_2.png \parameter{150}{0.5}{128}
Gold\_3.png \parameter{500}{0.5}{128}

Nach der Aufnahme dieser vier Bilder sind wir wieder auf die $MoS_2$-Probe umgestiegen:

\begin{table}[H]
\caption{MoS2\_5, \_6, \_7, \_8, \_9.png}
\centering \begin{tabular}[H]{l c c l}
Funktion & Approach Speed & Tip Voltage & Resultat/Bemerkung\\ \hline
Approach & 35\% & 0.15 V & orange - abgebrochen\\
 & 39\% & 0.15 V & grün\\
 & 26\% & 78 mV & grün
\end{tabular}
\end{table}

MoS2\_5.png \parameter{56.1}{0.5}{128}
MoS2\_6.png \parameter{6.13}{0.5}{128}
MoS2\_7.png \parameter{1.8}{0.5}{128}

Approach mit -36 mV und 35\% (grün)

MoS2\_8.png \parameter{1.8}{0.5}{128}

Approach mit -5.8 mV und 20\% (grün)

MoS2\_9.png \parameter{1.8}{0.5}{128}

Nach diesen Messungen haben wir eine neue Spitze gemacht und folgende Bilder mit Graphit aufgenommen:

\begin{table}[H]
\caption{Graphit\_F1, \_F2, \_F3, \_F4, \_F5.png}
\centering \begin{tabular}[H]{l c c l}
Funktion & Approach Speed & Tip Voltage & Resultat/Bemerkung\\ \hline
Approach & 44\% & 0.15 V & grün (Aufnahme von Graphit\_F1.png)\\
Approach & 31\% & 86 mV & grün (Aufnahme von Graphit\_F2.png)\\
Move & & & grün (Aufnahme von Graphit\_F3 und \_F4.png)
\end{tabular}
\end{table}

Graphit\_F1.png \parameter{5}{0.2}{128}
Graphit\_F2 und \_F3.png \parameter{5}{0.6}{256}
Graphit\_F4.png (Das Bild ist bei 2/3 des Bildes abgebrochen) \parameter{5}{1.2}{512}
Zoom in das Bild Graphit\_F4.png (x = 0.04 nm , y = 1.7 nm) \parameter{3.3}{1.2}{512}

Neuer Approach:

\begin{table}[H]
\caption{Graphit\_F6.png}
\centering \begin{tabular}[H]{l c c l}
Funktion & Approach Speed & Tip Voltage & Resultat/Bemerkung\\ \hline
Approach & 21\% & 86 mV & grün\\ 
Approach & 21\% & 51 mV & grün
\end{tabular}
\end{table}

Graphit\_F6.png: \parameter{5}{0.6}{256}

Da wir mit dieser Spitze so schöne Bilder bekommen konnten, haben wir auch gleich anhand des Messprogramms versucht, die Gitterkonstante des Graphits zu berechnen. Dazu haben wir die integrierte Längenmessfunktion benutzt. Wir haben immer mehrere Maxima gemessen und gemittelt:

\begin{center}
\begin{tabular}[H]{l c c c l}
Bild & Anzahl & gemessene Länge & Gitterkonstante & Messrichtung\\ \hline
Graphit\_F4.png & 8 & 1.913 nm & 2.391 $\mathring A$ & senkrecht\\
 & 8 & 1.866 nm & 2.333 $\mathring A$ & senkrecht\\
Graphit\_F3.png & 15 & 2.954 nm & 1.970 $\mathring A$ & senkrecht\\
 & 10 & 1.974 nm & 1.974 $\mathring A$ & senktecht\\
 & 10 & 2.313 nm & 2.313 $\mathring A$ & waagerecht\\
Graphit\_F5.png & 11 & 2.059 nm & 1.872 $\mathring A$ & waagerecht\\
 & 8  & 1.873 nm & 2.341 $\mathring A$ & senkrecht\\
 & 7 & 1.627 nm & 2.324 $\mathring A$ & senkrecht
\end{tabular}
\end{center}

Die Gitterkonstante die wir in diesem Fall messen sollten ist in der Theorie 2,456 $\mathring A$. Viele unserer gemessenen Werte liegen relativ nahe an diesem Wert. Die Ausnahmen haben mit verzerrzen Bildern zu tun. Vor allem hat man bei allen Bildern verschiedene Resultate für verschiedene Messrichtungen, was auf diese Verzerrung stark hindeutet.

Wir messen nochmal das Molybdändisulfid mit dieser Spitze:

\begin{table}[H]
\caption{MoS2\_F6, \_F7, \_F8.png}
\centering \begin{tabular}[H]{l c c l}
Funktion & Approach Speed & Tip Voltage & Resultat/Bemerkung\\ \hline
Approach & 44\% & -0.15 V & grün (Messung von MoS2\_F6 und \_F7.png)\\
 & 24\% & -84 mV & grün\\
 & 24\% & -49 mV & grün (Messung von MoS2\_F8.png
\end{tabular}
\end{table}

mit den Bildparametern

MoS2\_F6.png \parameter{5}{0.2}{128}
MoS2\_F7.png \parameter{5}{0.6}{256}
MoS2\_F6.png \parameter{5}{0.6}{256}

Auch hier haben wir wieder versucht die Gitterkonstante mit dem Messprogramm zu berechnen:

\begin{center}
\begin{tabular}[H]{l c c c l}
Bild & Anzahl & gemessene Länge & Gitterkonstante & Messrichtung\\ \hline
MoS2\_F7.png & 5 & 1.372 nm & 2.74 $\mathring A$ & waagerecht\\
MoS2\_F8.png & 9 & 2.430 nm & 2.70 $\mathring A$ & waagerecht\\
 & 10 & 1.724 nm & 1.724 $\mathring A$ & senkrecht\\
 & 2 & 0.4278 nm & 2.139 $\mathring A$ & senkrecht\\
\end{tabular}
\end{center}

%Auswertung ???

\clearpage








