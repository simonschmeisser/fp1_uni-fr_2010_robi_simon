\documentclass[a4paper,oneside]{scrartcl} %twocolumn,
\usepackage[utf8]{inputenc} %für MAC: applemac; für Windows: latin1 statt utf8
\usepackage[T1]{fontenc}
\usepackage[ngerman]{babel}

\usepackage{amsmath}
\usepackage{amsfonts}
\usepackage{amssymb}

\usepackage{mathptmx}
\usepackage{microtype}
\usepackage[nice]{nicefrac}

\usepackage{booktabs}
\usepackage{graphicx}
\usepackage{float}
\usepackage{hyperref}
\usepackage{wrapfig}


\setkomafont{captionlabel}{\upshape\bfseries}
\setkomafont{caption}{\itshape}


\title{Rastertunnelmikroskop}
\author{Robi Pedersen \and Simon Schmeißer}
\date{Versuchsdurchführung 16.09. - 17.09.2010}

\hypersetup{
pdftitle={Rastertunnelmikroskop},
pdfauthor={Robi Pedersen, Simon Schmeißer}}

\providecommand{\e}[1]{\ensuremath{\times 10^{#1}}}
\providecommand{\bild}[2]{
	\begin{figure}[H]
	\centering \includegraphics*[viewport= 5 225 322 520 , width = 0.6\textwidth]{messwerte/#1}
	\caption{#2} 
	\end{figure}
	}
\providecommand{\parameter}[3]{	\begin{center}
	\begin{tabular}[H]{l c}
	Image Width & #1 nm\\ Time/Line & #2 s\\ Points/Line & #3\\
	\end{tabular}
	\end{center}
	}
\providecommand{\zweid}[4]{
	\begin{figure}[H]
	\begin{minipage}{0.5\textwidth}
	\centering \includegraphics*[viewport= 5 528 322 825 , width = \textwidth]{messwerte/#1}
	\caption{#2}
	\end{minipage}
	\begin{minipage}{0.5\textwidth}
	\centering \includegraphics*[viewport= 5 528 322 825 , width = \textwidth]{messwerte/#3}
	\caption{#4}
	\end{minipage}
	\end{figure}
	}
\providecommand{\zweiunddreid}[1]{
	\begin{figure}[H]
	\begin{minipage}{0.5\textwidth}
	\centering \includegraphics*[viewport= 5 528 322 825 , width = \textwidth]{messwerte/#1}
	\end{minipage}
	\begin{minipage}{0.5\textwidth}
	\centering \includegraphics*[viewport= 5 5 322 300 , width = \textwidth]{messwerte/#1}	
	\end{minipage}
	\end{figure}
	}
	
\begin{document}
\begin{titlepage}
  \maketitle
  \vfill
  \thispagestyle{empty}
\end{titlepage}

\tableofcontents
\clearpage
%*****************************************************************

\section{Aufgabenstellung}

Ziel dieses Versuches ist es, sich mit den Prinzipien der Rastertunnelmikroskopie vertraut zu machen. Dafür sollen folgende Proben untersucht werden:
\begin{enumerate}
 \item Untersuchung eines mit Gold beschichteten Gitters.
      Veranschaulichung der streng delokalisierten Elektronen des Metalls. Eichung des Scan-Bereiches des Piezo-Stellelementes.
 \item Auflösung und Ausmessung der atomaren Oberflächenstruktur des Halbmetalls Graphit.
 \item Auflösung und Ausmessung der atomaren Oberflächenstruktur des Halbleiters $MoS_2$. Dabei soll zwischen den beiden unterschiedlichen Atomsorten unterschieden werden können.
\end{enumerate}
\section{Theoretische Grundlagen}

%Zerfallskaskaden
%Wirkungsweise von Szintillationszählern und Photomultipliern
%Zufällige Koinzidenzen

\subsection{$\alpha$-Zerfall}

Zerfällt ein sogenannter Mutterkern $_Z^AX$ in einen Tochterkern $_{Z-2}^{A-4}Y$ und ein $He^{2+}$-Ion so spricht man von $\alpha$-Zerfall. Das $He^{2+}$-Teilchen wird $\alpha$-Teilchen genannt. Die Emission von $\alpha$-Teilchen ist ein quantenmechanischer Prozess, der durch das Tunneln des $\alpha$-Teilchens durch die Potentialbarriere des Coulombpotentials des Kerns ermöglicht wird. $\alpha$-Teilchen zeigen ein diskretes Spektrum und sind monochromatisch.

\subsection{$\beta$-Zerfall}

Man unterscheidet beim $\beta$-Zerfall zwischen $\beta^+$- und $\beta^-$-Zerfall. Der $\beta^+$-Zerfall ist die Umwandlung eines Protons in ein Neutron mit Emission eines Positrons und eines Elektronenneutrinos:

$$ \beta^+: p^+ \rightarrow n^0 + e^+ + \nu_e $$

Der $\beta^-$-Zerfall hingegen, ist die Umwandlung eines Neutrons in ein Proton, unter Emission eines Elektrons und eines Elektron-Antineutrinos:

$$ \beta^-: n^0 \rightarrow p^+ + e^- + \bar \nu_e $$

Der $\beta^+$-Zerfall ist nur möglich für Protonen in einem Kern, da durch die Bindungsenergie die für den Prozess nötige Energie aufgebracht werden kann.

Als weitere Art des $\beta$-Zerfall zählt man noch den Elektroneneinfang (EC, \emph{electron capture)}. Dieser kann stattfinden, wenn ein Elektron der untersten Schale (K-Schale), wegen dessen Aufenthaltswahrscheinlichkeit im Kern, vom Kern eingefangen wird und mit einem Proton zu einem Neutron wird unter Emission eines Elektronenneutrinos:

$$ EC: p^+ + e^- \rightarrow n^0 + \nu_e $$

Der Elektroneneinfang gewinnt mit größeren Kernzahlen an Bedeutung, da dann die Aufenthaltswahrscheinlichkeit im Kern immer größer wird.

\subsection{$\gamma$-Strahlung}

Beim $\alpha$- und $\beta$-Zerfall gehen die Mutterkerne immer mit bestimmten Wahrscheinlichkeiten in verschiedene Zustände der Tochterkerne über. Letztere zerfallen innerhalb einer sehr kurzen Zeit ($\sim 10^-9 bis 10^-12 s$) in den Grundzustand und emittieren dabei ein Photon, genannt $\gamma$-Quant. Dieses $\gamma$-Quant wird entweder direkt vom Kern emittiert oder kann durch folgende Prozesse induzieren:

\subsubsection{Innere Konversion}

Die Energie des Kerns geht an ein Hüllenelektron, welches dadurch das Atom mit der verbleibenden Energie als kinetische Energie verlässt. Diese Elektronenlücke wird durch ein Elektron aus einer höheren Schale ersetzt, welches diesen Energieverlust durch ein Photon (bei schweren Elementen) oder den Auger-Effekt (bei leichten Elementen) kompensiert.

\subsubsection{Der Auger-Effekt}

Falls dieses oben erklärte Elektron dieses Loch füllt, kann es sein, dass die überschüssige Energie wiederum an ein Elektron abgegeben wird, welches dadurch das Atom verlässt und damit ein zweites Loch im Atom hinterlässt.


\subsection{Wechselwirkung von Photonen mit Materie}

\subsubsection{Das Absorptionsgesetz}

Ein Photonenstrahl der einfallenden Intensität $I_0$ nimmt exponentiell mit der Dicke x einer durchquerten Materieschicht an Intensität ab. Es gilt also:

$$ I = I_0\cdot e^{-\mu x} $$

wobei $\mu$ der mediumabhängige (und photonenenergieabhängige) Absorptionkoeffizient ist. $\mu$ lässt sich als Summe der Absorptionskoeffizienten aller möglichen Wechselwirkungen von Photonen mit Materie schreiben, nämlich des Photoeffekts, des Compton-Effekts und der Paarbildung.

$$\mu = \mu_{Ph.} + \mu_{C} + \mu_{PB} $$

\subsubsection{Der Photoeffekt}

\begin{figure}[H]
	\begin{minipage}{0.6\textwidth}
	\centering \includegraphics[width=\textwidth]{Bilder/Photoeffekt.png}
	\caption{Photoeffekt}
	\end{minipage}
	\begin{minipage}{0.4\textwidth}
	Wenn ein $\gamma$-Quant ein Elektron aus der Atomhülle herausschlägt, spricht man vom Photoeffekt. 	Dieser Effekt tritt also nur an gebundenen Elek\-tro\-nen auf. Die Energie des Photons geht zum größten Teil 	auf das Elektron, ein Teil wird jedoch als Rückstoßenergie vom Atom aufgenommen. Die 				Ab\-sorp\-tions\-wahrscheinlichkeit ist am größten in der K-Schale, also näher beim Kern. Der Photoeffekt 	dominiert gegenüber den anderen Effekten vor allem bei großen Atomen und Energien unter 100keV des	Photons. Nach dem Ablösen des Elektrons strahlt das ionisierte Atom die Bindungsenergie wieder ab.
	\end{minipage}
\end{figure}

\subsubsection{Der Compton-Effekt}

\begin{figure}[H]
	\begin{minipage}{0.4\textwidth}
	Trifft ein Photon auf ein leicht gebundenes oder freies Elektron, so wird das Photon nicht ganz absorbiert, sondern gibt einen Teil seiner Energie an das Elektron ab und wird selbst gestreut. Durch die Streuung verliert das Photon somit an Energie, d.h. die Frequenz des gestreuten Quants ist kleiner. Der Compton-Effekt dominiert bei Energien zwischen 100 keV und einigen MeV.
	\end{minipage}
	\begin{minipage}{0.6\textwidth}
	\centering \includegraphics[width=\textwidth]{Bilder/Comptoneffekt.png}
	\caption{Comptoneffekt}
	\end{minipage}

\end{figure}

\subsubsection{Die Paarbildung}

\begin{figure}[H]
	\begin{minipage}{0.5\textwidth}
	\centering \includegraphics[width=\textwidth]{Bilder/Paarbildung.jpg}
	\caption{Paarbildung}
	\end{minipage}
	\begin{minipage}{0.5\textwidth}
	Hat der $\gamma$-Quant mindestens die doppelte Ruheenergie eines Elektrons, also 1.022 MeV, so kann dieses im Feld eines Atomkerns (Stoßpartner für die Energie-Impuls-Erhaltung) ein Elektron-Positron-Paar erzeugen. Das Positron kann in Anwesenhait von Materie nicht frei existieren und wird abgebremst und annihiliert mit einem Elektron zu 2 bis 3 $\gamma$-Quanten. Im wahrscheinlicheren Fall zweier Quanten, erhalten diese jeweils eine Energie von 511 keV und bewegen sich in einem Winkel von 180$^\circ$ relativ zueinander (wegen Impulserhaltung).
	\end{minipage}
\end{figure}

\subsection{Der Szintillationszähler}

\begin{figure}[H]
	\centering \includegraphics[width=\textwidth]{Bilder/Szinti.png}
	\caption{Schema eines Szintillationszählers}
\end{figure}

Ein Szintillationszähler ist ein Gerät zum Nachweis ionisierender Strahlung. Die vom Quant abgegebene Energie wird in Licht verwandelt und von einem Photomultiplier verstärkt. Dieses Licht ist proportional zur Energie des Quants.

\begin{figure}[H]
	\begin{minipage}{0.45\textwidth}
	Der Szintillator selber ist ein NaI-Kristall mit Tl-Aktivatorzellen. Das Bändermodell erlaubt eine einfache Beschreibung des Verhaltens. Nämlich befinden sich bei tiefen Temperaturen alle Elektronen des Kristalls im sogenannten Valenzband. Absorbieren diese Elektronen energiereiche Strahlung, z.B. durch $\gamma$-Quanten (v.a. Photoeffekt), so werden diese angeregt und steigen auf höhere Energieniveaus. Reicht die Energie aus, so werden die Elektronen ins Leitungsband gehoben. Falls sie nicht groß genug ist um das Elektron vom Valenzband ins Leitungsband zu heben, so können sogenannte Exzitonen entstehen, lose gekoppelte Elektron-Loch-Paare (siehe Bild). Die Exzitonen, können sich genau so wie die Leitungsband-Elektronen frei im Kristall 
bewegen und können unter Emission eines 
	\end{minipage}
	\begin{minipage}{0.55\textwidth}
	\centering \includegraphics[width=\textwidth]{Bilder/Bandmodell.png}
	\caption{Bändermodell beim NaI:Tl$^+$}
	\end{minipage}
\end{figure}
Photons wieder in den Grundzustand zurückkehren. Eintreffende $\gamma$-Quanten haben Energien im Bereich von 0.1 bis 1 MeV, regen also mehrere hundert Elektronen gleichzeitig an. Wenn diese Elektron-Loch-Paare rekombinieren, entstehen neue Photonen, welche in den Photomultiplier geraten und dort verstärkt werden. Die Dotierung verformt lokal das Leitungsband und da diese Photonen vor allem an den Tl-Störstellen rekombinieren, reicht ihre Energie nicht aus um andere Elektronen ins Leitungsband anzuregen, somit können die Photonen nicht wieder vom Kristall absorbiert werden.






















\section{Versuchsbeschreibung}
\subsection{Pockelszelle}

\subsection{Faradayeffekt}
\section{Durchführung}


Wir haben den PC hochgefahren und dann das Messprogramm \emph{Nanosurf easyscan 2} gestartet. Dann haben wir eine Spitze anhand einer Kneifzange aus dem $PtIr$-Draht hergestellt und mit einer Pinzette in das Rastertunnelmikroskop eingeführt. Unsere erste Messung haben wir mit Graphit durchgeführt. Wir haben bei jeder Messung die Spitze grob an die Probe herangefahren. 

In diesem Fall haben wir für die feinere Annäherung wir folgende Schritte durchgeführt:



\begin{table}[H]
\caption{Graphit\_1.png}
\centering \begin{tabular}[H]{l c c l}
Funktion & Approach Speed & Tip Voltage & Resultat/Bemerkung\\ \hline
Approach & 52\% & 0.15 V & grün-rot blinkend\\
Withdraw & & & orange\\
Approach & 29\% & 0.15 V & grün\\
 & 21\% & 90 mV & grün\\
 & 20\% & 50 mV & grün\\
\end{tabular}
\end{table}


Wir haben dann ein Bild aufgenommen mit folgenden Parametern:

\parameter{1.5}{0.4}{256}

\bild{Graphit_1}{Graphit\_1.png}

Nach diesem ziemlich verfehlten Bild haben wir eine neue Spitze hergestellt und eingesetzt. Wir haben eine erneute Messung des Graphits durchgeführt: 

\begin{table}[H]
\caption{Graphit\_2.png}
\centering \begin{tabular}[H]{l c c l}
Funktion & Approach Speed & Tip Voltage & Resultat/Bemerkung\\ \hline
Approach & 50\% & 0.15 V & orange\\
 & 40\% & 0.15 V & rot\\
Retract & & & orange\\
Approach & 25\% & 0.15 V & orange\\
 & 20\% & 0.15 V & orange - abgebrochen\\
 & 27\% & 0.15 V & orange\\
 & 24\% & 0.15 V & orange\\
 & 23\% & 90 mV & orange - abgebrochen\\
 & 25\% & 90 mV & grün\\
 & 20\% & 50 mV & grün\\
 & 20\% & 25 mV & grün\\
\end{tabular}
\end{table}

Wir haben immer dann abgebrochen, wenn das Heranfahren der Nadel viel zu lange gedauert hat, also deutlich einige Minuten überschritten hat. Wir haben das Bild \emph{Graphit\_2.png} mit folgenden Parametern aufgenommen: 

\parameter{1.5}{0.4}{256}

In der Auswertung kann man alle relevanten Bilder einsehen. Wir beschreiben hier nur wie wir vorgegangen sind, um die Bilder aufzunehmen. Die Namen der Bilder entsprechen den angegebenen Namen über den Tabellen.

Wir haben in das Bild \emph{Graphit\_2.png} hineingezoomt und nach dem Feineinstellen nochmal eine genauere Messung durchgeführt. (x-Position = -1.3 nm , y-Position = -42 nm). Wir mussten hier nochmal die Nadel neu heranführen, da die Kontrolllampe orange war. 

\begin{table}[H]
\caption{Graphitmessung}
\centering \begin{tabular}[H]{l c c l}
Funktion & Approach Speed & Tip Voltage & Resultat/Bemerkung\\ \hline 
Approach & 46\% & 25 mV & rot\\
Withdraw & & & orange\\
Approach & 35\% & 25 mV & grün\\
\end{tabular}
\end{table}

\parameter{1.5}{0.8}{512}

An dieser Stelle haben wir wieder eine neue Spitze gemacht, da dieses Bild wieder sehr unklar war, was vielleicht daran liegen kann, dass wir beim ersten Approach die Probe mit der Spitze berührt haben (rote Kontrollleuchte), und somit vielleicht die Spitze deformiert haben. Wir haben es nochmal versucht:

\begin{table}[H]
\caption{Graphitmessung}
\centering \begin{tabular}[H]{l c c l} 
Funktion & Approach Speed & Tip Voltage & Resultat/Bemerkung\\ \hline
Approach & 46\% & 25 mV & rot\\
Withdraw & & & orange\\
Approach & 35\% & 25 mV & grün\\
\end{tabular}
\end{table}

\parameter{1.5}{0.6}{512}

und nochmal mit neuer Spitze:

\begin{table}[H]
\caption{Graphitmessung}
\centering \begin{tabular}[H]{l c c l} 
Funktion & Approach Speed & Tip Voltage & Resultat/Bemerkung\\ \hline
Approach & 40\% & 0.1 V & rot\\
Withdraw & & & orange\\
Approach & 30\% & 0.1 V & grün\\
 & 20\% & 25 mV & grün\\
\end{tabular}
\end{table}

\parameter{5.58}{0.6}{256}

Da alle diese Bilder ziemlich nutzlos waren sind wir an dieser Stelle aus frustrationstechnischen Gründen auf eine andere Probe umgestiegen, nämlich das 160-nm-Gitter (\emph{Nano-Grid}) und haben wiederum eine neue Spitze angefertigt. Das Heranführen der Nadel war jedoch erfolglos, da das Rastertunnelmikroskop keinen Tunnelstrom registrieren konnte. Wir haben somit unsere Messungen mit Molybdändisulfid fortgeführt.

\begin{table}[H]
\caption{MoS2\_1.png, MoS2\_2.png, MoS2\_3.png}
\centering \begin{tabular}[H]{l c c l} 
Funktion & Approach Speed & Tip Voltage & Resultat/Bemerkung\\ \hline
Approach & 50\% & 0.15 V & rot\\
Withdraw & & & orange\\
Approach & 30\% & 0.15 V & rot\\
Withdraw & & & orange\\
Approach & 20\% & 25 mV & grün\\
 & 20\% & 0.90 V & grün\\
\end{tabular}
\end{table}

Wir haben 3 Bilder mit jeweils verschiedenen Zooms bei dieser Einstellung gemacht. Die Bildparameter lauten:

MoS2\_1.png \parameter{351.4}{0.4}{128}
MoS2\_2.png, \ \ \ Zoom: x-Pos = -17 nm , y-Pos = -0.17 nm \parameter{37.06}{0.4}{128}
MoS2\_3.png, \ \ \ Zoom: x-Pos = -16 nm , y-Pos = -0.17 nm \parameter{12.31}{0.4}{128}

Nach einem erneuten vergeblichen Versuch, das Nano-Grid zu messen, haben wir eine weitere Messung des $MoS_2$ gemacht. Der Approach lief folgendermaßen ab:

\begin{table}[H]
\caption{Molybdändisulfidmessung}
\centering \begin{tabular}[H]{l c c l} 
Funktion & Approach Speed & Tip Voltage & Resultat/Bemerkung\\ \hline
Approach & 40\% & 0.15 V & rot\\
Withdraw & & & orange\\
Approach & 25\% & 0.1 V & orange-abgebrochen\\
 & 30\% & 0.1 V & orange-abgebrochen\\
 & 40\% & 0.1 V & grün\\
\end{tabular}
\end{table}

Nachdem die Messung begonnen hatte, wurde das gesamte Bild auf einmal weiß und die Kontrollleuchte zeigte rot an. Wir haben einen neuen Approach versucht:

\begin{table}[H]
\caption{MoS2\_4.png}
\centering \begin{tabular}[H]{l c c l}
Funktion & Approach Speed & Tip Voltage & Resultat/Bemerkung\\ \hline
Approach & 30\% & 0.1 V & orange - abgebrochen\\
 & 35\% & 0.1 V & grün
\end{tabular}
\end{table}

\parameter{10.2}{0.2}{128}

Mit einer neuen Spitze haben wir wiederum die Graphit-Probe analysiert:

\begin{table}[H]
\caption{Graphit\_3.png}
\centering \begin{tabular}[H]{l c c l}
Funktion & Approach Speed & Tip Voltage & Resultat/Bemerkung\\ \hline
Approach & 51\% & 0.19 V & rot\\
Withdraw & & & orange\\
Approach & 35\% & 0.15 V & rot\\
Withdraw & & & orange\\
Approach& 20\% & 0.1 V & rot\\
Withdraw & & & orange\\
Approach & 39\% & 0.1 V & grün\\
\end{tabular}
\end{table}

\parameter{1.96}{0.5}{256}

Nach dem weiteren Anfertigen einer neuen Spitze, haben wir nochmal die Graphit-Probe benutzt:

\begin{table}[H]
\caption{Graphit\_4.png}
\centering \begin{tabular}[H]{l c c l}
Funktion & Approach Speed & Tip Voltage & Resultat/Bemerkung\\ \hline
Approach & 36\% & 0.15 V & grün\\
 & 30\% & 90 mV & grün\\
 & 25\% & 60 mV & grün\\
 & 20\% & 45 mV & grün
\end{tabular}
\end{table}

\parameter{2}{0.5}{128}

Wir haben wiederum eine neue Spitze angefertigt und mit der Messung des Goldes begonnen. 

\begin{table}[H]
\caption{Gold\_0, \_1, \_2, \_3.png}
\centering \begin{tabular}[H]{l c c l}
Funktion & Approach Speed & Tip Voltage & Resultat/Bemerkung\\ \hline
Approach & 31\% & 0.1 V & grün\\
 & 21\% & 45 mV & grün\\
\end{tabular}
\end{table}

Gold\_0.png \parameter{3.5}{0.5}{128}
Gold\_1.png \parameter{3.5}{0.5}{128}
Gold\_2.png \parameter{150}{0.5}{128}
Gold\_3.png \parameter{500}{0.5}{128}

Nach der Aufnahme dieser vier Bilder sind wir wieder auf die $MoS_2$-Probe umgestiegen:

\begin{table}[H]
\caption{MoS2\_5, \_6, \_7, \_8, \_9.png}
\centering \begin{tabular}[H]{l c c l}
Funktion & Approach Speed & Tip Voltage & Resultat/Bemerkung\\ \hline
Approach & 35\% & 0.15 V & orange - abgebrochen\\
 & 39\% & 0.15 V & grün\\
 & 26\% & 78 mV & grün
\end{tabular}
\end{table}

MoS2\_5.png \parameter{56.1}{0.5}{128}
MoS2\_6.png \parameter{6.13}{0.5}{128}
MoS2\_7.png \parameter{1.8}{0.5}{128}

Approach mit -36 mV und 35\% (grün)

MoS2\_8.png \parameter{1.8}{0.5}{128}

Approach mit -5.8 mV und 20\% (grün)

MoS2\_9.png \parameter{1.8}{0.5}{128}

Nach diesen Messungen haben wir eine neue Spitze gemacht und folgende Bilder mit Graphit aufgenommen:

\begin{table}[H]
\caption{Graphit\_F1, \_F2, \_F3, \_F4, \_F5.png}
\centering \begin{tabular}[H]{l c c l}
Funktion & Approach Speed & Tip Voltage & Resultat/Bemerkung\\ \hline
Approach & 44\% & 0.15 V & grün (Aufnahme von Graphit\_F1.png)\\
Approach & 31\% & 86 mV & grün (Aufnahme von Graphit\_F2.png)\\
Move & & & grün (Aufnahme von Graphit\_F3 und \_F4.png)
\end{tabular}
\end{table}

Graphit\_F1.png \parameter{5}{0.2}{128}
Graphit\_F2 und \_F3.png \parameter{5}{0.6}{256}
Graphit\_F4.png (Das Bild ist bei 2/3 des Bildes abgebrochen) \parameter{5}{1.2}{512}
Zoom in das Bild Graphit\_F4.png (x = 0.04 nm , y = 1.7 nm) \parameter{3.3}{1.2}{512}

Neuer Approach:

\begin{table}[H]
\caption{Graphit\_F6.png}
\centering \begin{tabular}[H]{l c c l}
Funktion & Approach Speed & Tip Voltage & Resultat/Bemerkung\\ \hline
Approach & 21\% & 86 mV & grün\\ 
Approach & 21\% & 51 mV & grün
\end{tabular}
\end{table}

Graphit\_F6.png: \parameter{5}{0.6}{256}

Da wir mit dieser Spitze so schöne Bilder bekommen konnten, haben wir auch gleich anhand des Messprogramms versucht, die Gitterkonstante des Graphits zu berechnen. Dazu haben wir die integrierte Längenmessfunktion benutzt. Wir haben immer mehrere Maxima gemessen und gemittelt:

\begin{center}
\begin{tabular}[H]{l c c c l}
Bild & Anzahl N & gemessene Länge L & Gitterkonstante g & Messrichtung\\ \hline
Graphit\_F4.png & 8 & 1.913 nm & 2.391 $\mathring A$ & senkrecht\\
 & 8 & 1.866 nm & 2.333 $\mathring A$ & senkrecht\\
Graphit\_F3.png & 15 & 2.954 nm & 1.970 $\mathring A$ & senkrecht\\
 & 10 & 1.974 nm & 1.974 $\mathring A$ & senktecht\\
 & 10 & 2.313 nm & 2.313 $\mathring A$ & waagerecht\\
Graphit\_F5.png & 11 & 2.059 nm & 1.872 $\mathring A$ & waagerecht\\
 & 8  & 1.873 nm & 2.341 $\mathring A$ & senkrecht\\
 & 7 & 1.627 nm & 2.324 $\mathring A$ & senkrecht
\end{tabular}
\end{center}

Die Gitterkonstante die wir in diesem Fall messen sollten ist in der Theorie 2,456 $\mathring A$. Viele unserer gemessenen Werte liegen relativ nahe an diesem Wert. Die Ausnahmen haben mit verzerrzen Bildern zu tun. Vor allem hat man bei allen Bildern verschiedene Resultate für verschiedene Messrichtungen, was auf diese Verzerrung stark hindeutet. In der Auswertung haben wir die Fehler auf diese Werte berechnet.

Wir messen nochmal das Molybdändisulfid mit dieser Spitze:

\begin{table}[H]
\caption{MoS2\_F6, \_F7, \_F8.png}
\centering \begin{tabular}[H]{l c c l}
Funktion & Approach Speed & Tip Voltage & Resultat/Bemerkung\\ \hline
Approach & 44\% & -0.15 V & grün (Messung von MoS2\_F6 und \_F7.png)\\
 & 24\% & -84 mV & grün\\
 & 24\% & -49 mV & grün (Messung von MoS2\_F8.png
\end{tabular}
\end{table}

mit den Bildparametern

MoS2\_F6.png \parameter{5}{0.2}{128}
MoS2\_F7.png \parameter{5}{0.6}{256}
MoS2\_F6.png \parameter{5}{0.6}{256}

Auch hier haben wir wieder versucht die Gitterkonstante mit dem Messprogramm zu berechnen:

\begin{center}
\begin{tabular}[H]{l c c c l}
Bild & Anzahl N & gemessene Länge L & Gitterkonstante g & Messrichtung\\ \hline
MoS2\_F7.png & 5 & 1.372 nm & 2.74 $\mathring A$ & waagerecht\\
MoS2\_F8.png & 9 & 2.430 nm & 2.70 $\mathring A$ & waagerecht\\
 & 10 & 1.724 nm & 1.724 $\mathring A$ & senkrecht\\
 & 2 & 0.4278 nm & 2.139 $\mathring A$ & senkrecht\\
\end{tabular}
\end{center}

\clearpage









\section{Durchführung und Auswertung}


\clearpage
\section{Zusammenfassung}

Im ersten Teil des Versuchs haben wir das Magnetfeld des Permanentmagneten mit einer Hallsonde vermessen, um herauszufinden, an welchen Stellen das Magnetfeld homogen ist, und wie stark das Feld an dieser Stelle ist. Die Ermittlung der Homogenität haben wir bei den weiteren Messungen benutzt, die Magnetfeldstärke beträgt:

$$ \boxed{B = (434.1 \pm 4.5)\ mT} $$

Dann haben wir versucht die Resonanzfrequenz der 3 vorliegenden Proben zu messen, indem wir die Frequenz so verändert haben, dass die Absorptionslinien äquidistant waren. Diese Frequenz wurde dann gemessen und daraus der Landé-Faktor $g$ und das gyromagnetische Verhältnis $\gamma$ berechnet:


\begin{center}
\begin{tabular}{| l | c | c | c | c |} \hline
Probe & $\nu_r$ / MHz & $g$ & $\sigma(g)$ & $\gamma\ / MHz\cdot T^{-1}$\\ \hline
Glykol & $18.184 \pm 0.050$ & $5.495 \pm 0.059$ & 2 & $263.2 \pm 2.8$ \\
Wasserstoff &$18.180 \pm 0.002$ & $ 5.494 \pm 0.057 $ & 2 & $263.2 \pm 2.7$ \\
Teflon & $17.114 \pm 0.002$ &  $5.172 \pm 0.054$ & 2 & $247.7 \pm 2.6$\\ \hline
\end{tabular}
\end{center}

Für die Teflonprobe haben wir dann noch das kernmagnetische Moment $\mu_K$ berechnet, indem wir den Wert für $g$ als gegeben annahmen. Wir erhielten den Wert:

$$\boxed{\mu_K = (4.968 \pm 0.052)\cdot 10^{-27} J/T} $$

Der theoretische Wert liegt in der 2.Standard-Abweichung von diesem Wert.

Mithilfe der Wasserstoffprobe haben wir eine zweite Messung der Homogenität des Magnetfeldes durchgeführt und eine leichte lineare Abhängigkeit der Wirkung des Magnetfelds auf die Probe in Abhängigkeit von dessen Position auf dem vorhin ermittelten Plateau feststellen können.

Im letzten Teil des Versuchs haben wir anhand des Lock-In-Verfahrens durch einen linearen Fit noch ein Mal die Resonanzfrequenz der Wasserstoffprobe gemessen und erhielten:

$$\boxed{\nu_r = (18.190 \pm 0.004)\ MHz}$$

Dieser Wert liegt sehr dicht an der vorher ermittelten Resonanzfrequenz (0.055\%).
\section{Anhang - Protokoll}

\begin{appendix}
 
\begin{figure}[H]
\centering \includegraphics[width=\textwidth]{Bilder/Protokoll/001.png}
\end{figure}  
\begin{figure}[H]
\centering \includegraphics[width=0.97\textwidth]{Bilder/Protokoll/002.png}
\end{figure}  
\begin{figure}[H]
\centering \includegraphics[width=\textwidth]{Bilder/Protokoll/003.png}
\end{figure}  
\begin{figure}[H]
\centering \includegraphics[width=\textwidth]{Bilder/Protokoll/004.png}
\end{figure}
 
 
\end{appendix}


\clearpage

%\bibliographystyle{alphadin} 
%\bibliography{bib}


\end{document}
