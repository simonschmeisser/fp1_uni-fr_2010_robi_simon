\section{Auswertung}

Bei der Auswertung zeigen wir nur die Bilder, die uns tatsächlich relevante Aussagen über die Struktur machen können und die Bilder, die wir zur Messung der Gitterkonstante benutzt haben.

\subsection{Fehlerabschätzung auf die Gitterkonstante}

Wir haben bei der Gitterkonstante immer die Distanz über mehrere Maxima gemessen und gemittelt. Den Fehler auf diese Gesamtlänge haben wir mit $s_L = 0.5 \mathring A$ abgeschätzt, den Fehler auf die Anzahl zusätzlich mit $s_N = 1$, da diese nicht immer sehr klar war. Somit ergibt sich der Fehler auf die Gitterkonstante $g=L/N$ mit:

$$s_g = g\sqrt{\frac{s_L^2}{L^2} + \frac{s_N^2}{N^2}} $$

\subsection{Gold}

\subsubsection{Gold\_3.png ($\sim$ 500 nm)}
\zweiunddreid{Gold_3.png}
\subsubsection{Gold\_2.png ($\sim$ 150 nm)}
\zweiunddreid{Gold_2.png}
\subsubsection{Gold\_1.png (atomar)}
\zweiunddreid{Gold_1.png}

Wie man sehen kann ist die Auflösung beim Gold nicht besonders gut. Dies liegt vor allem daran, dass Gold ein Metall ist, und wie in der Theorie bereits vermerkt, die Atome delokalisiert sind, und wir somit keine atomare Auflösung erwarten können. Eine Messung der Gitterkonstanten konnten wir also nicht durchführen.

\subsection{Graphit}

\subsubsection{Graphit\_F3.png (atomar)}
\zweiunddreid{Freitag/Graphit_F3.png}

Bei diesem Bild haben wir eine Gitterkonstante waagerecht von 2.313 $\mathring{A}$ erhalten. Die beiden anderen, zur Messung der Gitterkonstanten benutzten Bilder sind folgende:

\zweid{Freitag/Graphit_F4.png}{Graphit\_F4.png}{Freitag/Graphit_F5.png}{Graphit\_F5.png}

Die (relevanten) Resultate sind 
\begin{table}{c c c}
	$g^{F4}_1 = 2.391 \mathring{A}$ (senkrecht) & und & $g^{F4}_2 = 2.391 \mathring{A}$ (senkrecht)\\
	$g^{F5}_1 = 2.341 \mathring{A}$ (senkrecht) & und & $g^{F5}_2 = 2.324 \mathring{A}$ (senkrecht)\\
\end{table}





2.391 $\mathring{A}$ bzw. 2.333 $\mathring{A}$ (F4, beide senkrecht) und 2.341 $\mathring{A}$ bzw. 2.324 $\mathring{A}$ (F5, beide senkrecht).

\subsection{Molybdändisulfid}

\subsubsection{MoS2\_5.png ($\sim$ 50 nm)}
\zweiunddreid{MoS2_5.png}
\subsubsection{MoS2\_F7.png (atomar)}
\zweiunddreid{Freitag/MoS2_F7.png}

Bei diesem Bild haben wir eine Gitterkonstante von 2.74 $\mathring{A}$ gemessen(waagerecht). Außerdem  haben wir für folgendes Bild noch eine Gitterkonstane von 2.70 $\mathring{A}$ gemessen (waagerecht):

\begin{figure}[H]
	\centering \includegraphics*[viewport= 5 528 322 825 , width = 0.5\textwidth]{messwerte/Freitag/MoS2_F8.png}
	\caption{MoS2\_F8.png}
	\end{figure}