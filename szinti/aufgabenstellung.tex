\section{Aufgabenstellung}
\begin{enumerate}
 \item Fertigen Sie ein Blockschaltbild des Versuchs an.
 \item Betrachten Sie den radioaktiven Zerfall von $^{152}Eu$.
	Vergleichen Sie die Signallängen des NaI-Szintiliationszählers mit denen des
	Plastikszintillationszählers. (Betriebsspannung NaJ: U = 625 V, Plastik: U $\sim$ 1900 V
	Betrachten Sie auf dem Oszilloskop die Signale des NaI-Szintiliationszählers
	nach dem
	\begin{enumerate}
	\item Photomultiplier
	\item Amplifier (unipolarer und bipolarer Ausgang)
	\item Single Channel Analyzer (SCA)
	\item Gate and Delay Generator
	\end{enumerate}
	Bestimmen Sie die zeitliche Verzögerung des Signals zwischen dem Eingang
	des Amplifiers und dem
	\begin{enumerate}
	 \item unipolaren Ausgang des Amplifiers
	 \item bipolaren Ausgang des Amplifiers
	 \item Ausgang des SCAs
	\end{enumerate}

	Beobachten und diskutieren Sie verschiedene Einstellungen der NIM-Geräte
	(Verstärkungsfaktoren, shaping time, ...). Finden Sie geeignete Einstellungen
	und stellen Sie die Signalverläufe in einem Diagramm wie in Abbildung 1 dar.
	Nehmen Sie die ( -Spektren der Präparate $^{22}Na$, $^{60}Co$ und $^{152}Eu$ mit einer Messdauer
	von jeweils 30 min auf. Nach einer Untergrundskorrektur wird anschließend anhand
	der bekannten Energien der intensiven Linien in den drei Spektren die Kanal-
	Energie-Eichung des Multi Channel Analyzers durchgeführt.
\item Nehmen Sie das (-Spektrum des RdTh-Präparates ($^{228}Th$) mit einer Mess-
dauer von 180 min auf. Bestimmen Sie nach einer Untergrundskorrektur des
Thoriumspektrums die Energien der auftretenden Linien und interpretieren Sie ihren
Ursprung. Bestimmen Sie für das $^{228}Th$-Präparat die gemessenen Intensitäten und
korrigieren sie mit Hilfe der Ausbeutekurve für den 3x3" NaI-Kristall. Vergleichen Sie
Ihre Ergebnisse mit den aus der Kemkartei berechneten Intensitäten.
\item Führen Sie eine Untergrundsmessung von 180 min durch. Bestimmen Sie die
Energie der auftretenden intensiven Linie und interpretieren Sie ihren Ursprung.
\item Bestimmen Sie das optimale Delay zwischen beiden Szintillationszählern.
Messen Sie mit diesem Delay bei $^{22}Na$ die "Winkelkorrelation" der zwei 511keV
Vernichtungsphotonen. Wählen Sie dazu eine sinnvolle Messdauer pro Winkel-
einstellung.
Berücksichtigen Sie durch eine weitere Messung die Anzahl zufälliger
Koinzidenzen.


\end{enumerate}
