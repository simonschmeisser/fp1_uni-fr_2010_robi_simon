\section{Durchführung und Auswertung}

\subsection{Einfluss der Einstellungen der NIM-Geräte auf das Signal}

Wir haben verschiedene Einstellungen der NIM-Geräte ausprobiert und deren Einfluss auf das Signal am Oszilloskop beobachtet. Als Signalquelle haben wir einfach die Umgebungsstrahlung genutzt und als Messgerät den Plastikszintillator.

Als erstes haben wir das Signal direkt am Vorverstärker gemessen. Danach haben wir den Einfluss der Verstärkung und der \emph{shaping time} am unipolaren Ausgang des Amplifiers auf das Signal untersucht. Schlussendlich haben wir noch den positiven Ausgang des Einkanalanalysators untersucht, indem wir die untere und die obere Schwelle eingestellt haben und die Funktion des Delays beobachtet haben.

Zusammengefasst erhielten wir folgende Signale:

\begin{figure}[H]
\begin{minipage}{0.4\textwidth}
\centering \includegraphics[width = \textwidth]{messergebnisse/1.JPG}
\end{minipage}
\begin{minipage}{0.6\textwidth}
Signal nach dem Vorverstärker.
\centering \begin{tabular}{l l}
Tiefe des Peaks & 5.8 mV\\
Breite des Peaks & 1.2 $\mu s$
\end{tabular}
\end{minipage}
\end{figure}

\begin{figure}[H]
\begin{minipage}{0.4\textwidth}
\centering \includegraphics[width = \textwidth]{messergebnisse/2.JPG}
\end{minipage}
\begin{minipage}{0.6\textwidth}
Signal nach dem Verstärker (unipolarer Ausgang) mit einer Verstärkung von 12.6 und einer shaping time von 0.5 $\mu s$

\centering \begin{tabular}{l l}
Spannung Peak to Peak & 36 mV\\
Zeit Peak to Peak & 1.3 $\mu s$
\end{tabular}
\end{minipage}
\end{figure}

\clearpage %----------------------

Einfluss höherer Verstärkung (shaping time: 0.5 $\mu s$)
\begin{figure}[H]
\begin{minipage}{0.5\textwidth}
\centering \includegraphics[width = 0.8\textwidth]{messergebnisse/3.JPG} 
\centering \begin{tabular}{l l}
Verstärkung & 100\\
Spannung Peak to Peak & 240 mV\\
Zeit Peak to Peak & 1.3 $\mu s$
\end{tabular}
\end{minipage}
\begin{minipage}{0.5\textwidth}
\centering \includegraphics[width = 0.8\textwidth]{messergebnisse/4.JPG}
\centering \begin{tabular}{l l}
Verstärkung & 500\\
Spannung Peak to Peak & 580 mV\\
Zeit Peak to Peak & 1.3 $\mu s$
\end{tabular}
\end{minipage}
\end{figure}

Die Verstärkung hat also wie erwartet die Amplitude des Signals vergrößert, z.B. haben wir bei einer Einstellung der Vergrößerung von 500 das Signal bereits auf eine Amplitude von 0.6 V bringen können, wo es am Anfang gerade mal etwa 6 mV hatte (100-fache Verstärkung). 

Der Einfluss der Shaping Time auf das Signal war schwer zu beobachten, da bei größerer Shaping Time das Signal immer instabiler wurde. Wir konnten jedoch sehen, dass das Signal breiter wurde. Bei einer eingestellten Shaping Time von 1 $\mu s$ konnten wir z.B. noch messen, dass die Distanz zwischen den beiden Peaks etwa 2.3 $\mu s$ betrug, im Gegensatz zu 1.3 $\mu s$ bei einer Shaping Time von 0.5 $\mu s$. Das Verstellen der Verstärkung hatte keinen sichtbaren Einfluss auf diese Distanz.

Danach haben wir das Signal des negativen Ausgangs des Einkanalanalysators untersucht und mit dem Signal aus dem Vorverstärker verglichen.

\begin{figure}[H]
\begin{minipage}{0.4\textwidth}
\centering \includegraphics[width = \textwidth]{messergebnisse/5.JPG}
\end{minipage}
\begin{minipage}{0.6\textwidth}
\centering \begin{tabular}{l l}
Verstärkung & 1000\\
Shaping Time & 0.5 $\mu s$\\
Tiefe des Signals & 1.6 V\\
Breite des Signals & 0.7 $\mu s$
\end{tabular}
\end{minipage}
\end{figure}

Die Distanz zwischen beiden Peaks ist hier etwa 2.2 $\mu s$. Durch Einstellen des Delays kann man diese ändern. Wir haben uns den Einfluss klar gemacht und folgende Werte gemessen:\\

\begin{center}
\begin{tabular}{l c c c c}
Delay & 0.12 & 3.0 & 7.4 & 10.0\\
$\Delta t / \mu s$ & 2.25 & 2.6 & 3.0 & 3.2
\end{tabular}
\end{center}

Die Erhöhung des Lower Levels hat dazu geführt, dass wir seltener ein Signal erhalten haben. Ist der Lower Level zu hoch, so erhält man kein Signal mehr. Ebenso erhält man kein Signal mehr, wenn man den Upper Level zu niedrig einstellt. Ist er zu hoch, so flackert das Signal.\\

Der positive Ausgang des Einkanalanalysators hat anstatt eines negativen Peaks, wie beim negativen Ausgang, einen Kasten mit positiver Amplitude herausgegeben. Bei einer Verstärkung von 1000 und einer Shaping Time von 0.5 $\mu s$ hatte der Kasten eine Amplitude von 6 V und eine Breite von 0.5 $\mu s$. Die Einstellungen, wie z.B. Verstärkung, Delay und Shaping Time hatten den gleichen Einfluss auf das positive wie auf das negative Signal.\\

Im Unterschied zum Plastikszintillator ist das Signal des NaI-Szintillators viel stärker. Die Shaping Time kann auch viel freier gewählt werden, ohne dass das Signal zu flackern beginnt. Z.B. haben wir bei einer Verstärkung von 30 und einer Shaping Time von 10 $\mu s$ ein Signal am unipolaren Ausgang mit einer Amplitude von etwa 7 V und einer Peak to Peak-Länge von 30$\mu s$ erhalten. Das Signal sieht anders aus als beim Plastikszintillator, nämlich hat man hauptsächlich einen großen, positiven Peak und dann noch einen kleinen negativen, der etwa 10 mal kleiner ist. Beim bipolaren Ausgang erhält man ein Signal mit 2 gleich großen Peaks und mit der oben genannten Einstellung erhielten wir die Peak to Peak-Amplitude von 12 V und eine Zeitdifferenz von 22 $\mu s$ zwischen den Peaks.

\subsection{Energieeichung des MCA}

Wir haben für die Messungen den NaI-Szintillator benutzt. Zur Energieeichung des Vielkanalanalysators haben wir die Spektren von 3 Proben aufgenommen, nämlich $^{22}Na$, $^{60}Co$ und $^{152}Eu$. Alle Messungen wurden mit der Verstärkung 20 und der Shaping Time 2 $\mu s$ aufgenommen.

Die bekannten Peaks der Spektren der 3 Proben wurden mit einer Gaußkurve gefittet. Aus diesen Peaks und den theoretischen Werten dieser Peaks, können wir nun die Energie-Kanal-Abhängigkeit ausrechnen. Die aus den Graphen ermittelten Werte sind folgende:\\

\begin{center}
\begin{tabular}{| l | c | c | c | c | c | c |} \hline
Präparat & $^{22}Na$ & $^{22}Na$ & $^{60}Co$ & $^{60}Co$ & $^{152}Eu$ & $^{152}Eu$\\ \hline
Energie / keV (theoretisch) & 511 & 1274 & 1173 & 1332 & 122 & 344\\ \hline
Kanal & 146.8 & 354.1 & 326.1 & 369.3 & 38.1 & 100.4\\ \hline
$\chi^2$/ndf & 1.24 & 1.70 & 0.82 & 1.14 & 65.40 & 5.09\\ \hline
\end{tabular}
\end{center}

Die theoretische Werte wurden aus der Staatsexamensarbeit entnommen. Den Fehler auf den Kanal schätzen wir 3. Auf den theoretischen Wert der Energie nehmen wir keinen Fehler an.

\begin{figure}
\centering \includegraphics[width = 0.85\textwidth]{auswertung/Na22.png}
\caption{$^{22}Na$}
\end{figure}

\begin{figure}[H]
\centering \includegraphics[width = 0.85\textwidth]{auswertung/Co60.png}
\caption{$^{60}Co$}
\end{figure}

\begin{figure}[H]
\centering \includegraphics[width = 0.85\textwidth]{auswertung/Eu152.png}
\caption{$^{152}Eu$}
\end{figure}


In dem wir den Kanal gegen die Energie auftragen, können wir per linearen Fit die Abhängigkeit dieser beiden Größen berechnen (siehe hierzu den Anhang). Als Steigung erhalten wir:

$$ b = (3.662 \pm 0.011) $$

und als Achsenabschnitt

$$ a = (-21.91 \pm 2.76) keV $$

Somit ist die Umrechnung vom Kanal K in Energie E:

$$ \boxed{E = bK + a = 3.662\cdot K - 21.91} $$ 
mit dem Fehler
$$ s_E = \sqrt{K^2s_b^2 + b^2s_k^2 + s_a^2} = \sqrt{0.000114\cdot K^2 + 128.30} $$

\clearpage %-----------------------------------------------

\subsection{Auswertung des Thoriums}

\begin{figure}[H]
\centering \includegraphics[width = \textwidth]{auswertung/Thoriumganz.png}
\caption{Untergrundkorrigiertes Spektum von $^{228}Th$}
\end{figure}

Die Thoriumprobe wurde 55530.307 s lang gemessen. Der Untergrund wurde hochgerechnet und dann vom Spektrum abgezogen. Wir erhalten das oben dargestellte Spektrum. Zur Auswertung fitten wir wieder Gaußkurven an die Spitzen und können dann mit der oben bestimmten Energie-Kanal-Abhängigkeit die Energie der einzelnen Peaks ausrechnen und dann interpretieren.

Wir erhalten folgende Kanäle und daraus folgende Energien:

\begin{center}
\begin{tabular}{| c | r | r | c |} \hline
Peak & Kanal & Energie E & $s_E$\\ \hline
1 & 26.74 & 76.0 & 11.3\\
2 & 46.30 & 147.6 & 11.3\\
3 & 70.91 & 237.7 & 11.4\\
4 & 78.67 & 266.1 & 11.4\\
5 & 99.18 & 341.2 & 11.4\\
6 & 117.04 & 406.6 & 11.4\\
7 & 146.33 & 513.9 & 11.4\\
8 & 166.23 & 586.8 & 11.5\\
9 & 233.66 & 833.7 & 11.6\\ \hline
\end{tabular}
\end{center}

Einen erwarteten 10. Peak von etwa 2.6 MeV konnten wir nicht sehen, da wir die Verstärkung so eingestellt haben, dass der Maximalwert für die Energie bei Kanal 512 nur 1.85 MeV war. Hier eine übersicht der gefitteten Peaks:

\begin{figure}[H]
\centering \includegraphics[width = \textwidth]{auswertung/Th1.png}
\caption{Erster Teil des $^{228}Th$-Spektrums}
\end{figure}

\begin{figure}[H]
\centering \includegraphics[width = \textwidth]{auswertung/Th2.png}
\caption{Zweiter Teil des $^{228}Th$-Spektrums}
\end{figure}

\subsection{Interpretation der Peaks des Thorium-228-Spektrums}

\begin{center}
\begin{tabular}{| c | r | c | p{6cm} | c |} \hline
Peak & E / keV & $E_{theo}$ / keV & Interpretation &St-Abw. \\ \hline
1 & 76.0 $\pm$ 11.3  & 84.37 & $^{228}Th \rightarrow ^{224}Ra$, wahrscheinlichster Übergang & 1 \\
2 & 147.6 $\pm$ 11.3 & 131.6 und 166.4 & $^{228}Th \rightarrow ^{224}Ra$, Überlagerung der beiden Peaks der theoretischen Energien & jeweils 2\\
3 & 237.7 $\pm$ 11.4 & 216.0 & $^{228}Th \rightarrow ^{224}Ra$ & 2\\
4 & 266.1 $\pm$ 11.4 & 241.0 und 238.6 & $^{224}Ra \rightarrow ^{220}Rn, \text{\ \ sowie \ \ } ^{212}Pb \rightarrow ^{212}Bi$, beide Peaks sind überlagert & jeweils 3\\
5 & 341.2 $\pm$ 11.4 & 300.1 & $^{212}Pb \rightarrow ^{212}Bi$ & 4\\
6 & 406.6 $\pm$ 11.4 & 453.0 & $^{212}Bi \rightarrow ^{208}Tl$ & 4\\
7 & 513.9 $\pm$ 11.4 & 510.8 & $^{208}Tl \rightarrow ^{208}Pb$ & 1\\
8 & 586.8 $\pm$ 11.5 & 583.2 & $^{208}Tl \rightarrow ^{208}Pb$ & 1\\
9 & 833.7 $\pm$ 11.6 & 804.9 & $^{216}Po \rightarrow ^{212}Pb$ & 3\\ \hline
\end{tabular}
\end{center}

\emph{St-Abw.} gibt die Anzahl der Standardabweichungen an, in denen der theoretische Wert der Energie von dem gemessenen liegt. Unsere Werte liegen also alle in einem guten Bereich, nahe genug an den erwarteten Werten. Angegeben bei der Interpretation ist immer der Zerfall des Mutterkerns in den Tochterkern, welcher dann ein $\gamma$-Quant der genannten Energie emittiert, um in den Grundzustand über zu gehen.


\subsection{Interpretation der auftretenden intensiven Linie des Untergrunds}

\begin{figure}[H]
\centering \includegraphics[width = \textwidth]{auswertung/UG1.png}
\caption{Der gemessene Untergrund (120 min)}
\end{figure}

Wir untersuchen den etwas intensiveren Peak bei Kanal 400. Nach einem Gauß-Fit erhalten wir einen Wert für den Kanal von 403.58. Für die Energie folgt somit:

$$E = (1456.0 \pm 12.2) keV $$

Wir interpretieren dies als ein Kalium-40-Peak ($E_{theo} = 1460.8$, 1 Standard-Abweichung), da der Gauß-Fit sehr dicht an dem Wert liegt, und da Kalium in der Umgebung vorkommt.

\begin{figure}[H]
\centering \includegraphics[width = \textwidth]{auswertung/UG2.png}
\caption{Zoom auf Kanäle 350-450 des Untergrunds}
\end{figure}

\subsection{Winkelverteilung der $^{22}Na$ Vernichtungsphotonen}

Bevor wir mit der Messung begonnen haben wir die Levels (upper und lower level) angepasst, so dass nur die $\gamma$-Quanten gemessen werden, die eine Energie von 511 keV haben. Außerdem mussten anhand des Delays beide Szintillatoren so eingestellt werden, dass sie zeitgleiche Signale herausgeben, damit die Koinzidenzen auch tatsächlich gemessen werden konnten.\\

Zur Koinzidenzmessung haben wir dann den Plastikszintillator gegenüber vom NaI-Szintillator gedreht, und zwar um einen Winkel von 90$^\circ$ bis 200$^\circ$. Es wurden jedes mal 120 s gemessen. Nach den Messungen haben wir noch eine Messung ohne Probe gemacht, bei der wir 5 counts erhalten haben. Dieser Untergrund (zufällige Koinzidenzen) wurde von den ermittelten Werten abgezogen.\\

Der Fehler auf die Counts (Fehlerbalken) berechnet sich durch $s_N = \sqrt{N}$. Den Fehler auf den Einzelwert haben wir mit $s_\alpha = 0.5^\circ$ abgeschätzt.

\begin{figure}[H]
\centering \includegraphics[width = \textwidth]{auswertung/Winkel.png}
\caption{Winkelverteilung der Vernichtungsphotonen}
\end{figure}

Das Maximum liegt also bei
$$\alpha = (181.48 \pm 0.40)^\circ$$
Wir konnten somit die theoretische Vorhersage von 180$^\circ$ bestätigen. Dieser Wert liegt in der 4. Standardabweichung. Wir gehen bei der kleinen Abweichung eher von einem systematischen Fehler aus, welcher jedoch sehr klein ist.

Messtabelle (mit Untergrundkorrektur)

\begin{center}
\begin{tabular}{c c c c c c c c}
90$^\circ$ & 100$^\circ$ & 110$^\circ$ & 120$^\circ$ & 130$^\circ$ & 140$^\circ$ & 150$^\circ$ & 160$^\circ$\\ \hline
7 & 4 & 8 & 12 & 17 & 15 & 12 & 7 \\
& & & & & & & \\
165$^\circ$ & 170$^\circ$ & 175$^\circ$ & 180$^\circ$ & 185$^\circ$ & 190$^\circ$ & 195$^\circ$ & 200$^\circ$\\ \hline
11 & 15 & 92 & 161 & 160 & 38 & 16 & 14\\
\end{tabular}
\end{center}

























