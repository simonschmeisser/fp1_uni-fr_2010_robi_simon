\section{Zusammenfassung}

Im ersten Teil des Versuchs haben wir uns mit den Einstellmöglichkeiten der NIM-Geräte vertraut gemacht und deren Einfluss auf das Ausgangssignal eines Szintillators untersucht. Untersucht wurden: Verstärkung, Shaping Time, unipolarer und bipolarer Ausgang, Einkanalanalysator (lower level, upper level, delay, positiver und negativer Ausgang), sowie der Multikanalanalysator.\\

Dann haben wir anhand von bekannten Spektren die Kanäle des Multikanalanalysators festen Energie zuordnen können und haben den linearen Zusammenhang

$$E = bK + a = (3.662 \pm 0.011)\cdot K - (21.91 \pm 2.76)$$

berechnen können. Diese Eichung haben wir dann benutzt um die Energien der Peaks des Thorium-228-Zerfallsspektrums zu ermitteln. Wir haben 9 Peaks mit einer Gaußkurve gefittet und die Energie aus dem Kanal ausgerechnet und diese dann interpretiert.

\begin{center}
\begin{tabular}{| c | r | c | p{6cm} | c |} \hline
Peak & E / keV & $E_{theo}$ / keV & Interpretation &St-Abw. \\ \hline
1 & 76.0 $\pm$ 11.3  & 84.37 & $^{228}Th \rightarrow ^{224}Ra$, wahrscheinlichster Übergang & 1 \\
2 & 147.6 $\pm$ 11.3 & 131.6 und 166.4 & $^{228}Th \rightarrow ^{224}Ra$, Überlagerung der beiden Peaks der theoretischen Energien & jeweils 2\\
3 & 237.7 $\pm$ 11.4 & 216.0 & $^{228}Th \rightarrow ^{224}Ra$ & 2\\
4 & 266.1 $\pm$ 11.4 & 241.0 und 238.6 & $^{224}Ra \rightarrow ^{220}Rn, \text{\ \ sowie \ \ } ^{212}Pb \rightarrow ^{212}Bi$, beide Peaks sind überlagert & jeweils 3\\
5 & 341.2 $\pm$ 11.4 & 300.1 & $^{212}Pb \rightarrow ^{212}Bi$ & 4\\
6 & 406.6 $\pm$ 11.4 & 453.0 & $^{212}Bi \rightarrow ^{208}Tl$ & 4\\
7 & 513.9 $\pm$ 11.4 & 510.8 & $^{208}Tl \rightarrow ^{208}Pb$ & 1\\
8 & 586.8 $\pm$ 11.5 & 583.2 & $^{208}Tl \rightarrow ^{208}Pb$ & 1\\
9 & 833.7 $\pm$ 11.6 & 804.9 & $^{216}Po \rightarrow ^{212}Pb$ & 3\\ \hline
\end{tabular}
\end{center}

Das Spektrum des Untergrunds enthielt einen intensiveren Peak, den wir als Kalium-40-Peak deuten konnten. Unsere erhaltene Energie war 

$$E = (1456.0 \pm 12.2) keV $$

und die theoretische Energie des Kalium-40-Peaks ist: $E_{theo} = 1460.8 keV$.\\

Im letzten Teil haben wir dann die Winkelverteilung der Na-22-Vernichtungsphotonen nachgewiesen, indem wir beide Szintillatoren gegeneinander gedreht haben. Wir hielten ein Maximum der Koinzidenzen bei

$$\alpha = (181.48 \pm 0.40)^\circ$$

was der Theorie entspricht.