\section{Versuchsbeschreibung}

Ziel dieses Versuches ist es, die Halbwertszeit einiger Stoffe zu ermitteln, die sehr langsam zerfallen. Dabei handelt es sich um Samarilium (\atom{147}{}{Sm}), einen $alpha$-Strahler und Kalium (\atom{40}{}{K}), einen $beta$-Strahler.

\subsection{Labview-Programm}
Im ersten Versuchsteil soll ein LABVIEW-Programm erstellt werden, dass die Hochspannungsquelle des Zählrohres ansteuert, Ereignisse aufnimmt und die Zählrate berechnet.

\subsection{Zählrohrcharakteristik}
Es soll die Elektronik optimiert werden um dann die Charakteristik des verwendeten Zählrohrs aufzunehmen. Als Probe wird dabei Uran (\atom{238}{}{U}) verwendet. Weiterhin soll der Untergrund, also die Zählrate mit einem leeren Probenhalter, ermittelt werden.

\subsection{Bestimmung der Halbwertszeit von \atom{147}{}{Sm} ($\alpha$-Zerfall)}
Es wird zuerst der gesamte Spannungsbereich vermessen, in dem wir nach der Uran-Messung den Proportionalitätsbereich für den $\alpha$-Zerfall vermuten. Dadurch ermitteln wir die Spannung $U_{\alpha}$ in der Mitte des Plateaus sowie die ungefähre Zählrate $n$ bei dieser. Über
\begin{align}
 \frac{S_N}{N} \stackrel{!}{} 0.02 & \Leftrightarrow \frac{1}{\sqrt{N}} = \frac{1}{\sqrt{n \cdot t}} = 0.02 \\
				   & \Leftrightarrow n \cdot t \cdot 0.0004 = 1 \\
				   & \Leftrightarrow t = \frac{1}{0.0004 \cdot n}\\
\end{align}
können wir berechnen, wie lange mindestens gemessen werden muss, damit der relative Fehler unter 2 \% sinkt. Entsprechend wird dann eine solche Messung bei $U_{\alpha}$ durchgeführt.

Mit den Gleichungen \ref{t12ln2lambda} und \ref{dNdtlambdaN} erhalten können wir dann die Halbwertszeit $t_{1/2}$ bestimmen:
\begin{align}
 A & = \lambda N \\
 A & = \frac{\ln 2}{t_{1/2}}N \\
 t_{1/2} & = \ln 2 \frac{N}{A} \\
\end{align}

hierbei entspricht $N$ der Anzahl der Atome und $A$ der Aktivität, also der Anzahl der Zerfälle. Die Anzahl der Atome ist hier allerdings die Anzahl der Atome, deren Zerfallsprodukte auch tatsächlich registriert werden und nicht bereits in der Probe wieder eingefangen werden. Daher kann sie nicht einfach über Masse und Dichte berechnet werden, sondern wir verwenden
