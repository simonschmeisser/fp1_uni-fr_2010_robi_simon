\section{Zusammenfassung}

Es gelang uns mit einer \atom{238}{}{U}-Probe das Verhalten des Zählrohres aufzunehmen und grob die passenden Messbereiche für die Messung von $\alpha$- und $\beta$-Zerfällen zu ermitteln. 

\paragraph{Halbwertszeit von \atom{147}{}{Sm}} Wir konnten die Halbwertszeit von \atom{147}{}{Sm} näherungsweise bestimmen.
$$
 T_{1/2} \left( \atom{147}{}{Sm} \right) = (1.260 \pm 0.026) \cdot 10^11 a
$$

Unser Ergebniss liegt also 8 Standartabweichungen über dem Literaturwert $T_{1/2 - lit} \left( \atom{147}{}{Sm} \right) = 1.06 \cdot 10^11 a$.

\paragraph{Halbwertszeit von \atom{40}{}{K}} Die Halbwertszeit von \atom{40}{}{K} konnten wir ebenfalls bestimmen:
$$
  T_{1/2} \left( \atom{40}{}{K} \right) = ( 0.730 \pm 0.151 ) \cdot 10^9 a
$$

Der Literaturwert für die Halbwertszeit beträgt $T_{1/2} \left( \atom{40}{}{K} \right) = 1.277 \cdot 10^9 a$. Unser Ergebnis liegt also innerhalb von vier Standardabweichungen zum Literaturwert.

Bei diesem Versuch waren starke Schwankungen im Messsignal auffällig, weit über den Rahmen der statistisch Schwankungen hinaus. Diese führten dazu, dass viele Messungen entweder mehrfach durchgeführt werden mussten oder komplett nicht verwertbar waren. Bei erneuter Durchführung würden wir dazu neigen kurze bis sehr kurze Messzeiten zu wählen, da der statistische Fehler gegenüber dem systematischen verschwindend gering ist.

