\section{Theoretische Grundlagen}

\subsection{$\alpha$-Zerfall}
Als $\alpha$-Zerfall bezeichnet man den Zerfall eines Atoms, bei dem ein \atom{4}{2}{He} Ion den Kern verlässt. Die Reaktionsgleichung lautet also:
$$ \atom{A}{Z}{X} \rightarrow \atom{A-4}{Z-2}{Y}^{-2} + \atom{4}{2}{He}^{+2} $$

\atom{4}{2}{He} besitzt sowohl eine voll besetzte Neutronen- als auch eine voll besetzte Protononenschale, man bezeichnet es auch als "`doppelt magisch"', und hat somit eine sehr hohe Bindungsenergie. Klassisch betrachtet kann ein solches Ion den Atomkern dennoch nicht verlassen, da seine Energie niedriger ist als die des Coulombwalls. Quantenmechanisch ist dies natürlich durch den Tunneleffekt durchaus möglich. Die Wahrscheinlichkeit für einen $\alpha$-Zerfall ergibt sich aus dem Produkt der Einzelwahrscheinlichkeiten für die Bildung eines solchen Ions, seine Rate der "`Wallberührung"` sowie schließlich der Tunnelwahrscheinlichkeit:
$$ W = W_0 \times W_1 \times T $$
Die Tunnelwahrscheinlichkeit $T$ kann mit den Gamow-Faktor $G$ näherungsweise analytisch berechnen.

Die Energie des $\alpha$-Teilchens ergibt sich aus der Energiedifferenz zwischen Mutteratom und Tochteratom. Diese unterscheidet sich bei Zerfällen des selben Isotops höchstens durch angeregte Kernzustände. Beim Übergang vom angeregten Mutteratom zum Grundzustand des Tochteratoms erhält das Ion die höchste Energie, beim Übergang von Grundzustand zu Grundzustand ist sie entwas niedriger und für den Übergang von Grundzustand zu angeregtem Tochteratom ist sie am niedrigsten.

Da es also nur endlich viele diskrete Übergangsmöglichkeiten gibt, ist auch das Energiespektrum des $\alpha$-Zerfalls diskret.  

\subsection{$\beta$ - Zerfall}
Als $\beta$-Zerfall bezeichnet man den Übergang eines Protonen zu einem Neutronen oder eines Neutrons zu einem Proton. Der $\beta^-$-Zerfall wird beschrieben durch:
$$ \atom{A}{Z}{X} \rightarrow \atom{A}{Z+1}{Y} + e^- + \overline{\nu} $$
also
$$ n \rightarrow p + e^- + \overline{\nu_e} $$
Der $\beta^+$-Zerfall ist der Zerfall eines Protons in ein Neutron:
$$p \rightarrow n + e^+ + \nu_e$$

\subsection{Elektroneneinfang}

Als weitere Art des $\beta$-Zerfall zählt man noch den Elektroneneinfang (EC, \emph{electron capture)}. Dieser kann stattfinden, wenn ein Elektron der untersten Schale (K-Schale), wegen dessen Aufenthaltswahrscheinlichkeit im Kern, vom Kern eingefangen wird und mit einem Proton zu einem Neutron wird unter Emission eines Elektronenneutrinos:

$$ EC: p^+ + e^- \rightarrow n^0 + \nu_e $$

Der Elektroneneinfang gewinnt mit größeren Kernzahlen an Bedeutung, da dann die Aufenthaltswahrscheinlichkeit im Kern immer größer wird.


\subsection{$\gamma$-Strahlung}

Beim $\alpha$- und $\beta$-Zerfall gehen die Mutterkerne immer mit bestimmten Wahrscheinlichkeiten in verschiedene Zustände der Tochterkerne über. Letztere zerfallen innerhalb einer sehr kurzen Zeit ($\sim 10^-9 bis 10^-12 s$) in den Grundzustand und emittieren dabei ein Photon, genannt $\gamma$-Quant. Dieses $\gamma$-Quant wird entweder direkt vom Kern emittiert oder kann durch folgende Prozesse induzieren:

\subsubsection{Innere Konversion}

Die Energie des Kerns geht an ein Hüllenelektron, welches dadurch das Atom mit der verbleibenden Energie als kinetische Energie verlässt. Diese Elektronenlücke wird durch ein Elektron aus einer höheren Schale ersetzt, welches diesen Energieverlust durch ein Photon (bei schweren Elementen) oder den Auger-Effekt (bei leichten Elementen) kompensiert.

\subsubsection{Der Auger-Effekt}

Falls dieses oben erklärte Elektron dieses Loch füllt, kann es sein, dass die überschüssige Energie wiederum an ein Elektron abgegeben wird, welches dadurch das Atom verlässt und damit ein zweites Loch im Atom hinterlässt.

\subsection{Wechselwirkung von geladenen Teilen mit Materie}




- Wechselwirkung von geladenen Teilchen mit Materie --> Szinti
- Absorption und Reichweite radioaktiver Strahlung --> z.T. in Szinti
- Gasionisationsdetektoren
- Umgang mit radioaktiven Präparaten
























