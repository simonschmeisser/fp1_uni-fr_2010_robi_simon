\section{Theorie}

\subsection{Der Mitführungskoeffizient}

Die Geschwindigkeit des Lichtes in einem ruhenden Medium wird durch die Formel $$v=\frac{c}{n}$$ berechnet, wobei $c = 3\cdot 10^8 m/s$ die Lichgeschwindigkeit im Vakuum und $n$ der Brechungsindex des jeweiligen Mediums ist. Hat das Medium eine Geschwindigkeit w, so ergibt sich folgende Formel für die Geschwindigkeit des Lichtes:
\begin{equation} v = \frac{c}{n} \pm \alpha\cdot w  \end{equation}
$\alpha$ ist hier der Mitführungskoeffizient und es gilt:
\begin{equation} \alpha = 1 - \frac{1}{n^2} - \frac{\lambda\cdot\Delta n}{n \cdot\Delta\lambda} \end{equation}
Fresnel postulierte 1818, dass dieser die Mitführung des Äthers durch die Bewegung des Körpers beschreibt, daher der Name. Er leitete den Wert von $\alpha = 1 - 1/n^2$ her. Nach der experimentellen Widerlegung der Äthertheorie leiteten modernere Theorien, wie die Relativitätstheorie, diesen Faktor genauer her, sodass die oben genannte Formel (2) entstand.

\subsubsection{Herleitung des Mitführungskoeffizienten}

Die Relativitätstheorie besagt, dass wenn ein System $K'$ sich relativ zu einem System $K$ mit der Geschwindigkeit $w$ in x-Richtung bewegt, dass die Orts- und Zeitkoordinaten dieser Systeme durch die Lorentztransformationen zusammenhängen: 

\begin{equation} x' = \gamma (x-wt) \end{equation}
\begin{equation} t' = \gamma \left(t-\frac{wx}{c^2}\right) \end{equation}
\begin{equation} \text{mit \ } \gamma = \frac{1}{\sqrt{1-\frac{w^2}{c^2}}} \end{equation}

Außerdem gilt $y'=y$ und $z'=z$. Aus diesen Formeln ergibt sich die Formel für die relati\-vistische Addition von Geschwindigkeiten:

\begin{equation} v = \frac{v' + w}{1 + \frac{v'\cdot w}{c^2}} \end{equation}

(v und v' sind hier in x-Richtung). Für ein Lichtstrahl der Geschwindigkeit $v' = c/n$ und mit den Näherung $w<<c$ und somit $(1+a)^{-1}\approx 1-a$ und $\frac{w}{c} \approx 0$ erhält man schlussendlich die Fresnelsche Mitführungsformel:

\begin{equation} v\approx \frac{c}{n} \pm \left(1-\frac{1}{n^2}\right)w \end{equation}

Im Fall von der rotierenden Quarzscheibe gilt $w = w_x = w_r = w_i / n$  (siehe hierzu Abbildung 2 in 2.1). Außerdem betrachten wir nun auch die Abhängigkeit des Brechungsindexes eines Mediums von der Frequenz des gebrochenen Lichts, also $n = n(\nu)$. Nämlich folgt aus der Dopplerverschiebung

$$\nu' = \nu\left( 1\mp \frac{w_i}{c}\right)$$

die lineare Näherung für den Brechungsindex

$$n(\nu') = n(\nu) + (\nu' - \nu)\frac{dn}{d\nu} = n(\nu)\left(1 \mp \frac{w_i\cdot\nu}{c\cdot n(\nu)}\frac{dn(\nu)}{d\nu} \right)$$

und somit die Formel für die Geschwindigkeit, nachdem Näherungen gemacht wurden \\ ($w_r << c$):

\begin{align} 
v & \approx \frac{c}{n} \pm w_r\cdot\left(1 - \frac{1}{n^2} - \frac{w_i}{w_r}\frac{\nu}{n^2}\frac{dn}{d\nu} \right)\\
& = \frac{c}{n} \pm w_r\cdot\left(1 - \frac{1}{n^2} - \frac{w_i}{w_r}\frac{\lambda}{n^2}\frac{dn}{d\lambda} \right)
\end{align}

Bei der Quarzscheibe ergibt sich dann wegen $w_i / w_r = n$ schlussendlich:

\begin{equation}
\boxed{v = \frac{c}{n} \pm \alpha\cdot w = \frac{c}{n} \pm \left(1 - \frac{1}{n^2} - \frac{\lambda}{n}\frac{dn}{d\lambda} \right)w_r}
\end{equation}

Diese Formel brauchen wir, um den theoretischen Wert für $\alpha$ zu berechnen (siehe Aufgabenstellung). Den experimentellen Wert erhalten wir durch die Formel:

\begin{equation} \alpha = \frac{L\cdot\lambda\cdot\Delta\nu}{2\cdot n\cdot\omega\cdot d \cdot x_0} \end{equation}

wobei $L$ die optische Länge des Ringlasers, $d$ die Dicke der Quarzscheibe, $\omega$ die Drehfrequenz der Scheibe und $x_0$ der Auftreffpunkt des Lasers auf die Scheibe.


\subsection{Funktionsprinzip des Ringlasers}

Der Ringlaser ist ein dreieckiger Resonator, der aus 3 Spiegeln zusammengesetzt ist. In einem der drei Arme befindet sich ein He-Ne-Laser, der eine monochromatische, linear polarisierte Welle mit der Wellenlänge $6328.2 \ \mathring A$ in beide Richtungen in den Resonator abstrahlt. Somit läuft die ein Teil des Lichts links herum und ein Teil rechts herum (r-Strahl und l-Strahl). In einem zweiten Arm des Resonators befindet sich eine rotierende Quarzglasscheibe auf die der Laser unter dem Brewsterwinkel $\theta_B$ eintrifft, um Reflexionsverluste zu minimieren (siehe \emph{3.3. Der Brewserwinkel}). Durch die Rotation dieser Scheibe, wird das Licht innerhalb der Scheibe in die eine Richtung schneller, in die andere Richtung langsamer, jeweils um $+\alpha w_r$ oder $-\alpha w_r$. Hieraus folgt eine Änderung des optischen Weges und somit der optischen Länge des Resonators, sodass die Frequenzen in beide Richtungen erhöht bzw. erniedrigt werden. Hinter dem Auskoppelspiegel $S_3$ des Resonators interferieren diese Wellen dann und es entsteht eine Schwebung, aus der dann die Differenz der beiden Frequenzen ermittelt werden kann.


\subsection{Der Brewsterwinkel}





























