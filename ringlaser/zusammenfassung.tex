\section{Zusammenfassung}

Wir haben mit einem Ringlaser den Mitführungskoeffizienten $\alpha$ mit zwei verschiedenen Methoden gemessen und mit dem theoretischen Wert nach der Formel von Lorentz verglichen.

Als erstes haben wir den Mittelpunkt der Quarzscheibe berechnet, bei dem der Laser nicht "mitgeführt" wird. Der Wert hierfür lautet:

$$\bar x = (46,377 \pm 0,001) mm$$

Bei der ersten Methode haben wir die Quarzscheibe im Ringlaser mit konstanter Periode gedreht und den Durchtritspunkt des Lasers durch die Scheibe variiert. Dies haben wir für 4 feste Perioden gemacht. Die so ermittelten Werte für $\alpha$ haben wir mit Wichtung gemittelt und den folgenden Wert erhalten:

$$\alpha_1 = 0,5319 \pm 0,0001$$

Bei der zweiten Methode haben wir nun den Durchtrittspunkt insgesamt vier mal festgesetzt und die Periode der Quarzscheibe für jeden Messpunkt verändert. Für den Mitführungskoeffizienten haben wir erhalten:

$$\alpha_2 = 0,5427 \pm 0,0001$$

Der theoretische Wert nach der Formel von Lorentz lautet:

$$\alpha_{theo} = 0,5420$$

Somit weicht $\alpha_1$ um 1,9\% und $\alpha_2$ um 0,1\% von dem theoretischen Wert ab. Wir konnten also die Theorie mit unserem Experiment sehr gut nachweisen. Die sehr kleinen Fehler auf unsere experimentellen Werte sind dadurch zu begründen, dass wir starke Abweichungen als falsch betrachtet und somit  verworfen haben.

\clearpage