\section{Zusammenfassung}

Wir haben vor dem Versuch das SQUID aufgebaut und konnten das SQUID-Pattern gut optimieren. Dann haben wir Messungen mit einer stromdurchflossenen ringförmigen Leiterschleife mit fünf verschiedenen Widerständen durchgeführt. Diese Messungen haben wir jeweils mit dem theoretisch berechneten Wert verglichen und haben folgende Werte erhalten:\\

\begin{minipage}{0.5\textwidth}
\begin{center}
\begin{tabular}[H]{| c | c | c |} \hline
 & $B_z / nT$ berechnet & $B_z / nT$ aus dem Fit\\ \hline \hline
 $R_1$ & $1,696 \pm 0,219$ & $2,567\pm 0,003$\\
 $R_2$ & $1,022 \pm 0,132$ & $1,333\pm 0,004$\\
 $R_3$ & $0,376 \pm 0,048$ & $0,448\pm 0,001$\\
 $R_4$ & $0,227 \pm 0,029$ & $0,267\pm 0,001$\\
 $R_5$ & $0,117 \pm 0,015$ & $0,121\pm 0,002$\\ \hline
 \end{tabular}
 \end{center}
 \end{minipage} 
\begin{minipage}{0.5\textwidth}
\begin{center}
\begin{tabular}[H]{| c | c | c |} \hline
 & $p$ in $A\cdot mm^2$ ber. & $p$ in $A\cdot mm^2$ Fit\\ \hline \hline
 $R_1$ & $0,369 \pm 0,068$ & $0,599\pm 0,100$\\
 $R_2$ & $0,283 \pm 0,041$ & $0,311\pm 0,052$\\
 $R_3$ & $0,088 \pm 0,015$ & $0,105\pm 0,017$\\
 $R_4$ & $0,053 \pm 0,009$ & $0,062\pm 0,010$\\
 $R_5$ & $0,027 \pm 0,005$ & $0,028\pm 0,005$\\ \hline
 \end{tabular}
 \end{center}
 \end{minipage}\\
 
 Bei $R_4$ und $R_5$ liegen die gemessenen Werte jeweils in der 1. Standardabweichung, bei $R_3$ und $R_2$ in der 2. bzw 3. und der gemessene Wert für $R_1$ liegt in der 4.Standardabweichung vom berechneten. Die gemessenen Werte liegen also relativ dicht an den berechneten Werten, kleine Abweichungen kann man jedoch z.B. dadurch erklären, dass die Leiterschleife keine perfekte Ringform hat und somit die Berechnung nicht genau stimmt.
 
Im zweiten Teil des Versuchs haben wir weitere Proben anhand des SQUIDs gemessen und konnten somit Aussagen über deren magnetische Feldstärke und Dipolmoment machen. Die Tabelle der Messwerte lässt sich im Teil 4.3 einsehen und die Funktionen und Polarplots in Anhang C.