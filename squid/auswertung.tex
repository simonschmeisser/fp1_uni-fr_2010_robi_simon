\section{Auswertung}

Bevor wir das SQUID eingeschaltet haben, haben wir die Distanz der Probe zum SQUID-Sensor gemessen. Diese beträgt:

$$z = (3,6 \pm 0,2)\ cm$$

Nach dieser Messung haben wir den flüssigen Stickstoff in das Kryostat gefüllt und den SQUID-Sensor eingefügt. Wir haben 15 Minuten gewartet, und dann mit der Justierung begonnen.

\subsection{Justierung des SQUIDs}

Die Justierung des SQUIDs haben wir anhand des Programms \emph{JSQ Duo Sensor Control} an einem PC und einem Oszilloskop durchgeführt. Wir haben die Software in den Test-Modus geschaltet, welcher einen Funktionsgenerator dazu veranlasst, eine Dreieckspannung in den Schwingkreis einzukoppeln. Diese beeinflusst dann die Spannungsantwort des SQUID, die wir am Oszilloskop einsehen können, das sogenannte \emph{SQUID-Pattern}. Zum Triggern schließen wir dann den Funktionsgenerator noch direkt an den Oszilloskop.  Dann haben wir den Arbeitspunkt anhand der Software justiert, sodass die Amplitude des SQUID-Patterns maximal wurde. Unsere Einstellparameter lauten:\\

\begin{center}
\begin{tabular}[H]{l l}
	$VCA = 1391$ & Stromamplitude des Schwingkreises ($\sim 6\cdot 10^{-8} W$)\\
	$VCO = 1477$ & Frequenz des Schwingkreises ($\sim 760 MHz$)\\
	$OFF = 1471$  & Offset des Signals
\end{tabular}
\end{center}

\begin{figure}[H]
	\centering \includegraphics[width = 0.78\textwidth]{Bilder/Pattern1.png}
	\caption{Das SQUID-Pattern}
\end{figure}

\subsection{Dipolmoment und Magnetfeldstärken der Leiterschleife}

\subsubsection{Berechnung}

Wir haben eine stromdurchflossene Leiterschleife mit Durchmesser $d = (3,5 \pm 0,3)\ mm$ am SQUID für 5 verscheide Widerstände gemessen. Aus dem gemessenen Durchmesser ergibt sich der Radius $r=d/2$ mit Fehler $s_r = s_d/2$: $$r = (1,75 \pm 0,15) mm$$

Das magnetische Dipolmoment berechnen wir mit der Formel:

$$p = I\cdot A = \frac{U}{R}\pi r^2 \text{ \ \ \ mit \ \ \ } s_p = p\sqrt{\frac{s_U^2}{U^2} + \frac{s_R^2}{R^2} + 2\frac{s_r^2}{r^2}}$$

Die Feldstärke berechnen wir mit der Formel:

$$B_z = \frac{\mu_0}{2\pi}\frac{p}{z^3} \text{ \ \ \ mit \ \ \ } s_{B_z}=B_z\sqrt{\frac{s_p^2}{p^2} + 3 \frac{s_z^2}{z^2}}$$

Wir erhalten die Werte:

\begin{figure}[H]
	\centering \includegraphics{Bilder/Tab-Leiterschleife.jpg}
\end{figure}

\subsubsection{Graphisch}

Zur graphischen Berechnung der Dipolmomente und Feldstärken haben wir die Messwerte des SQUIDS sinusförmig gefittet nach der Funktion:

$$ f(A,B,C,D,x) = A + B\cdot\sin(C\cdot x + D) $$

Hier ist somit A der Offset, B die Amplitude, C die Frequenz und D die Phasenverschiebung. Wir erhalten die Ergebnisse:

\begin{tabular}[H]{| c | c | c | c | c | c |} \hline
 & $\chi^2 / ndf$ & Offset $A$ in $V$ & Amplitude $B$ in $V$ & Frequenz C in $^\circ/s$ & Phase $D$ \\ 
\hline
$R_1$ & 0.82 & -0.6252$\pm$0.0005 & 0.5244$\pm$0.0006 & 49.950$\pm$0.005 & $2.167 \pm 0.002$\\
$R_2$ & 1.64 & 0.6340$\pm$0.0006 & 0.2722$\pm$0.0009 & 49.916$\pm$0.014 & $-2.682 \pm 0.007$\\
$R_3$ & 0.22 & -0.5370$\pm$0.0002 & 0.0916$\pm$0.0003 & 49.936$\pm$0.015 & $3.208 \pm 0.007$\\
$R_4$ & 0.20 & -0.5263$\pm$0.0002 & 0.0546$\pm$0.0003 & 50.048$\pm$0.023 & $0.178 \pm 0.001$\\
$R_5$ & 0.32 & -0.6587$\pm$0.0003 & 0.0247$\pm$0.0004 & 49.953$\pm$0.071 & $3.032 \pm 0.035$\\ \hline
\end{tabular}

Mit der Formel 

$$ B_z = F \frac{\Delta V}{2s_i} = F\frac{B}{s_i} $$

berechnen wir anhand der Fits die Feldstärke. $F = 9.3 nT / \Phi_0$ ist der Feld-Fluss-Koeffizient, der Wert ist Herstellerangabe. $s_i$ ist der eingestellte Wert des Feedback-Resistors und beträgt in allen Fällen $1,9 V/ \Phi_0$. Der Fehler auf $B_z$ berechnet sich durch:

$$ s_{B_z} = B_z\frac{s_B}{B} $$

(B ist hier die aus den Fits bestimmte Amplitude der Sinusfunktion)
 
Die erhaltenen Werte haben wir in einer Tabelle zusammengefasst, um den direkten Vergleich mit den berechneten Werten zu erleichtern:

\begin{center}
\begin{tabular}[H]{| c | c | c |} \hline
 & $B_z / nT$ berechnet & $B_z / nT$ aus dem Fit\\ \hline \hline
 $R_1$ & $1,696 \pm 0,219$ & $2,567\pm ...$\\
 $R_2$ & $1,022 \pm 0,132$ & $1,333\pm ...$\\
 $R_3$ & $0,376 \pm 0,048$ & $0,449\pm ...$\\
 $R_4$ & $0,227 \pm 0,029$ & $0,268\pm ...$\\
 $R_5$ & $0,117 \pm 0,015$ & $0,121\pm ...$\\ \hline
 \end{tabular}
 \end{center}
 
 
 \subsection{Dipolmoment und Feldstärke anderer Proben}



















