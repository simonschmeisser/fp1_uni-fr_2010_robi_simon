\section{Theoretische Grundlagen}

\subsection{Supraleitung}

Supraleiter sind Materialien mit der Eigenschaft, dass sie widerstandslos Strom leiten können, wenn sie unter eine bestimmte Temperatur (genannt: kritische Temperatur $T_c$) gekühlt werden. Sie verhalten sich wie ideale Diamagneten, erzeugen also ein durch die inneren Ströme induziertes Magnetfeld, welches Magnetfelder von außen komplett kompensieren und somit sind sie im Inneren komplett feldfrei (Meißner-Ochsenfeld-Effekt). Je nach Verhalten in einem externen Magnetfeld unterscheidet man 2 Arten von Supraleiter:

\paragraph{Typ I: }  Das Magnetfeld wird bis auf eine dünne Schicht (exponentieller Abfall von außen, genannt \emph{London'sche Eindringtiefe}) an der Oberfläche des Leiters komplett aus dem Inneren verdrängt. Erreicht das Magnetfeld einen kritischen Wert $B_c$, so wird der Supraleiter wieder normalleitend. Die kritischen Temperaturen liegen beim Typ I bei maximal $23,2K$ ($Nb_3Ge$).

\paragraph{Typ II: } Supraleiter vom Typ II, auch Hochtemperatursupraleiter genannt, besitzen zwei kritische Magnetfelder $B_{c1}$ und $B_{c2}$ (mit $B_{c1} < B_{c2}$). Unter dem kritischen Feld $B_{c1}$ hat der Supraleiter die gleichen Eigenschaften wie ein Supraleiter des Typs I und über $B_{c2}$ ist er normalleitend. Zwischen beiden kritischen Feldstärken bilden sich im Supraleiter 'Flussfäden' aus, welche von Wirbelströmen umschlossen sind, in welchen das Magnetfeld verschieden von Null ist, und somit das Material normalleitend ist. Die anderen Stellen im Material sind jedoch noch supraleitend.

\subsection{BCS-Theorie}

Die von John Bardeen, Leon Cooper und John Schrieffer entwickelte BCS-Theorie versucht die Phänomene der Supraleitung zu erklären. Die Theorie besagt, dass in einem Supraleiter der Strom nicht durch freie Elektronen getragen wird, sondern durch sogenannte \emph{Cooper-Paare}: Dadurch, dass die Ionen in einem Atomgitter viel träger als die Elektronen sind, bildet sich hinter einem Elektron eine schwache positive Polarisation aus, da die Atome eine gewisse Zeit brauchen um wieder in ihre Ausgangsposition zurückzukehren. Diese Kraft wirkt anziehend auf andere Elektronen. Über mehrere Gitterkonstanten hinweg ist die resultierende Kraft aus dieser Kraft und der Coulomb-Abstoßung zwischen zwei Elektronen positiv. Beide Elektronen besitzen also eine Art Nettoanziehungskraft, die sie bindet. Diese Bindung kann man als Phononen-Wechselwirkung zwischen den Elektronen betrachten. Dieses Elektronenpaar kann man nun als ein Boson betrachten, da der Gesamtspin ganzzahlig ist. Dies führt dazu, dass alle Cooper-Paare den gleichen Zustand annehmen können und somit auch den Grundzustand. Dies ist energetisch günstiger und lässt sich als den ganzen Festkörper überspannende Bose-Einstein-Wellenfunktion beschreiben. Diese Wellenfunktion kann von lokalen Hindernissen wie z.B. Atomkernen nicht mehr beeinflusst werden, d.h. es entsteht keine Wechselwirkung mehr mit dem Rest vom Metall  und so ist der Ladungstransport widerstandslos.