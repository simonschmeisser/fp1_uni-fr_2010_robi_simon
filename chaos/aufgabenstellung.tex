\section{Aufgabenstellung}

\begin{enumerate}
\item Messung am linearen Schwingkreis

Messen Sie die Amplitude des Spannungsabfalls am Kondensator in 
Abhängigkeit von der Frequenz der treibenden Spannung. Bestimmen Sie durch 
einen Fit an die Daten den Widerstand, die Induktivität und die Kapazität der 
Schaltung und vergleichen Sie diese mit den auf den Bauteilen aufgedruckten 
Werten. Diskutieren Sie das Ergebnis. 

\item Messungen an verschiedenen nicht linearen Schwingkreisen 

\begin{enumerate}

\item Messen Sie die Amplitude des Spannungsabfalls an der Diode in 
Abhängigkeit von der Frequenz der treibenden Spannung. Achten sie     
darauf, dass die Amplitude der treibenden Spannung klein ist, damit        
keine Bifurkationen auftreten. Vergleichen Sie mit dem linearen          
Schwingkreis. 
\item Beobachten Sie den sogenannten "Weg ins Chaos" und erzeugen Sie 
durch Variation der Amplitude der treibenden Spannung Bifurkationen 
bzw. Periodenverdopplungen. Nutzen Sie dabei verschiedene 
Darstellungsmöglichkeiten mit dem Oszilloskop, z.B. x-y-Darstellung, mit 
der treibenden Spannung als x und dem Spannungsabfall an der Diode 
als y. Bestimmen Sie die ersten Folgenglieder an und dn, die gegen die 
Feigenbaumkonstanten a und Ö konvergieren. 
\item Bestimmen Sie die erste Wiederkehrabbildung für mindestens zwei 
 System und vergleichen Sie mit der logistischen Abbildung. Welche 
Gemeinsamkeiten, welche Unterschiede gibt es? 
\item Stellen Sie auf dem Schirm des Oszilloskops ein Amplitudenbifurka- 
tionsdiagramm dar. Nehmen Sie diese mit der digitalen Kamera auf, die 
beim Assistenten erhältlich ist und vergleichen Sie diese mit dem 
Bifurkationsdiagramm der logistischen Abbildung.

\end{enumerate}
\end{enumerate}