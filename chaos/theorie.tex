\section{Theoretische Grundlagen}

\subsection{Linearer Schwingkreis}
Ein elektromagnetischer Schwingkreis besteht aus einer Kapazität (üblicherweise einem Kondensator) und einer Induktivität (einer Spule). Diese können entweder in Reihe oder Parallel geschaltet werden. Für einen ungetriebenen Serienschwingkreis gilt die Differentialgleichung
\begin{equation}
 L \frac{d^2 I}{d t^2} + R \cdot \frac{d I}{d t} + \frac{1}{C} \cdot I = 0
\end{equation}
Nun verwendet man den Lösungsansatz
\begin{equation}
 I = A \cdot e^{\lambda \cdot t}
\end{equation}
und erhält
\begin{equation}
 \lambda^2 + \frac{R}{L} \cdot \lambda + \frac{1}{LC} = 0
\end{equation}





\subsection{Nichtlinearer Schwingkreis}

\subsection{Chaos}