<<<<<<< HEAD
\section{Theoretical Principles}

\subsection{Deterministic Chaos}

A chaotic system is a dynamical system, which is highly sensitive to its initial conditions. Changing them even by a very small difference leads to a completely different behavior of the system. Thus, a long-term prediction of the behavior is impossible. Nonetheless chaotic systems can be deterministic, meaning that no random elements are involved, but they still have unpredictable behavior.

\subsubsection{Bifurcation}

A periodic system usually has a circular trajectory in phase space and is defined by a single sharp frequency. In chaotic systems, this depends on a so called control parameter. By changing this parameter, the system can suddenly be overlaid by a second periodic oscillation with smaller amplitude, so that the resulting system has the double frequency of the initial system. This phenomenon is called a bifurcation. The circle in phase space gets back to its initial point after two cycles and in the spectrum shows two frequencies, the one of the initial system, and one which has half its frequency. By changing the control parameter even more, the system shows more and more of these bifurcations, until it turns into chaos, which means, that an infinite number of subharmonic oscillations overlay the original system. The resulting period of the system is infinite.

\subsubsection{The bifurcation diagram}

If, in the phase space of a chaotic system through bifurcations (as mentioned above), we lay a Poincaré-section, a hypersection perpendicular to the trajectories, at any point, and we plot those points against the control parameter, we obtain a so called bifurcation diagram, which contains informations about when unstable behavior takes place, and how long it lasts.

\begin{figure}[H]
\centering \includegraphics[width=0.8\textwidth]{Bilder/TheoBifDiag.png}
\caption{Representative Example of a Bifurcation Diagram}
\end{figure}

\subsubsection{Feigenbaum-Constants}

Every system, whose path to chaos can be described by bifurcations, underlies a certain universality. This universality is reflected in the Feigenbaum-Constants $\alpha$ and $\delta$.
Let's call $L_n$ the point of the nth bifurcation of our system. The Feigenbaum-Constant $\delta$ is then given by
\begin{equation} \delta = \lim_{n\to \infty} \delta_n = \lim_{n\to \infty}\ \frac{L_{n+1}-L_n}{L_{n+2}-L_{n+1}} = 4.6692016\dots \end{equation} and it has that value for every chaotic system.

\begin{figure}[H]
\begin{minipage}{0.4\textwidth}
The other Feigenbaum-Constant is called $\alpha$. It is defined by \begin{equation} \alpha = \frac{|d_n|}{|d_{n+1}|}=2.502907875\dots\end{equation} The value for $d_n$ is given by the distance from the intersection point of half the amplitude (figure 2 : x=1/2) and the bifurcation diagram to the next fixpoint on the diagram. $\alpha$ is the same for any two consecutive bifurcations and any bifurcation diagram.
\end{minipage}
\begin{minipage}{0.6\textwidth}
\includegraphics[width=\textwidth]{Bilder/alpha.png}
\caption{How to calculate $\alpha$}
\end{minipage}
\end{figure}

\subsubsection{An example: The Logistic Map}

The logistic map is defined by $$x_{i+1}=r\cdot x_i(1-x_i)$$ with $r$ as the control parameter. It has two fixpoints: $$x_0 = 0 \text{\ and \ } x_1 = 1-\frac{1}{r}$$ By changing the control parameter $r$, the logistic map bifurcates. Here is a table describing its behavior in function of $r$:\\

\begin{tabular}{c l}
$0 < r < 1$ & $x_0$ is the only (stable) fixpoint\\
$1 < r < 3$ & $x_0$ is unstable and $x_1$ ist a stable fixpoint\\
$3 < r < 4$ & there is no more stable fixpoint\\
$4 < r$ & the system is completely chaotic
\end{tabular}

\begin{figure}[H]
\centering \includegraphics[width=0.8\textwidth]{Bilder/logmap.png}
\caption{Bifurcation-Diagram of the Logistic Map}
\end{figure}

%FENSTER IM CHAOS

\subsection{The linear oscillating circuit}
\begin{figure}[H]
\begin{minipage}{0.5\textwidth}
\includegraphics[width=\textwidth]{Bilder/lincirc.png}
\caption{Diagram of a RLC-circuit}
\end{minipage}
\begin{minipage}{0.5\textwidth}
With Kirchhoff's law we get the relation \begin{eqnarray*}  
U_{ext} &=& U_0\cdot\sin\omega t \\
&=& U_L + U_R + U_C\\
&=& L\cdot \dot I + R\cdot I + Q/C
\end{eqnarray*}
With the relation $I = \dot Q$ we obtain the differential equation:
$$ L\ddot Q + R \dot Q + \frac{1}{C}Q = U_0\cdot\sin \omega t $$
\end{minipage}
\end{figure}

Using the Ansatz $Q(t) = Q_0 \cdot\cos(\omega t)$ we get:

$$Q(t) = \frac{U_{ext}}{-\omega^2L+i\omega R + 1/C}$$

and from this follows, that

$$U(\omega) = \frac{ \frac{U_0}{LC} }{ \sqrt{ (\frac{1}{LC} - \omega^2 )^2+ (\frac{R}{L}\omega)^2}}$$

We use this formula to fit the data of the linear oscillating circle:

\begin{equation} U(\omega) = \sqrt{\frac{A^2}{(B^2-4\pi^2f^2)^2 + (2\pi Cf)^2}} \end{equation}

with  $A=\frac{U_0}{LC}$ ,  $B=\frac{1}{\sqrt{LC}}$ ,  $C=\frac{R}{L}$

\subsection{The nonlinear oscillating circle}
\begin{figure}[H]
\begin{minipage}{0.5\textwidth}
\includegraphics[width=\textwidth]{Bilder/nlcirc.png}
\caption{Diagram of a RLC-circuit}
\end{minipage}
\begin{minipage}{0.5\textwidth}
In order to make a nonlinear oscillating circuit, we use a diode instead of a capacitor. If in such a circuit, the amplitude of the driving tension is increased, the system bifurcates until its behavior is completely chaotic. With the conditions
$$U_{ext} = L\dot I + R\dot I + U_{diode}$$
$$I_{ext}-I_{diode} = \dot Q = C(U)\dot U$$
we get the differential equation for the circuit:
\end{minipage}
\end{figure}

\begin{equation} 
\ddot Q + \left(\frac{R}{L} + \frac{1}{C(Q)} \frac{\partial I_{diode}}{\partial U}\right) + \frac{1}{L}(RI_{diode}+U)=U_{ext} \end{equation}

$C(U) = \frac{dQ}{dU}$ represents the capacitance of the diode.


=======
\section{Theoretische Grundlagen}

\subsection{Linearer Schwingkreis}
Ein elektromagnetischer Schwingkreis besteht aus einer Kapazität (üblicherweise einem Kondensator) und einer Induktivität (einer Spule). Diese können entweder in Reihe oder Parallel geschaltet werden. Für einen ungetriebenen Serienschwingkreis gilt die Differentialgleichung
\begin{equation}
 L \frac{d^2 I}{d t^2} + R \cdot \frac{d I}{d t} + \frac{1}{C} \cdot I = 0
\end{equation}
Nun verwendet man den Lösungsansatz
\begin{equation}
 I = A \cdot e^{\lambda \cdot t}
\end{equation}
und erhält
\begin{equation}
 \lambda^2 + \frac{R}{L} \cdot \lambda + \frac{1}{LC} = 0
\end{equation}
>>>>>>> f481230abd1fc5cc65ebbfbce615dc3a508d55a7





<<<<<<< HEAD

=======
\subsection{Nichtlinearer Schwingkreis}

\subsection{Chaos}
>>>>>>> f481230abd1fc5cc65ebbfbce615dc3a508d55a7
