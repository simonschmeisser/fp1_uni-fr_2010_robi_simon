\clearpage
\section{Conclusion}

In the first part we analyzed the behavior of the tension at a capacitor in an RLC-linear oscillating electrical circuit, by varying the input frequency of a frequency generator. We calculated the theoretical frequency an compared it to the measured one, by fitting the tension-frequency-dependence with a Lorentz-Peak. Here is a table of our results:

\begin{center}
\begin{tabular}{l c c c c}
Circ. & $f_{res,theo}/Hz$ & $f_{res,fit}/Hz$ & deviation & \#st. dev.\\ \hline
1 & $269.0\pm 3\cdot10^{-5}$ & $256.05\pm1.14$ & 5\% & 12\\
2 & $603.03\pm0.01$ & $585.91\pm1.71$ & 3\% & 11\\
3 & $11224.02\pm0.01$ & $11130.89\pm5.46$ & 0.8\% & 11\\
\end{tabular}
\end{center}

Afterwards we did the same measurement to find out the resonance frequency of a nonlinear oscillating circle with a diode instead of the capacitor. We found out, that it was 

$$f=(157.1\pm13.1)kHz$$

We then set the frequency fixed to (almost) the resonance frequency and tried to see the dependence of the system's behavior of the amplitude of the input tension. The periodical system would start to be overlapped by subharmonic oscillations. The higher the amplitude, the more subharmonic overlaps were to see. We then finally tried to set up the bifurcation diagram, but were unsuccessful. After retrying it with a new circuit with a resonance frequency of $f_{res}=97.619 kHz$ we were able to draw the bifurcation diagram and calculate the Feigenbaum-Constants:

\begin{center}
\begin{tabular}{c c c c}
experimental value & theoretical value & deviation & \#st. dev.\\ \hline
$\alpha = 4.00\pm0.55 $ & $\alpha = 2.50$ & 60\% & 3\\
$\delta = 4.28 \pm68.62$ & $\delta = 4.67$ & 9\% & 1\\
\end{tabular}
\end{center}

Finally, we modulated the input sinusoidal tension from the frequency generator with different functions, namely a triangular, a sawtooth and a lower frequency sine-function and switched to the x-y-mode of the oscillator to be able to see the bifurcation diagrams, which we were able. Using the sawtooth-function and the reverse sawtooth-function, we were able to see hysteresis effects.