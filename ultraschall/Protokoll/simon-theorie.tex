\subsection{Ultraschallzelle}
Die Ultraschallzelle nach Karolus und Helmberger \cite{Karolus} besteht aus zwei Schwingquarzen, welche entgegengesetz schwingen und dadurch eine stehende Welle erzeugen. Diese wird dann von Isooktan übertragen, wobei sich dessen Dichte periodisch ändert. Dadurch ändert sich der Brechungsindex lokal \cite{Raman}: 
\begin{equation}
 \mu \left( x \right) = \mu_0 - \mu sin \frac{2 \pi x}{ \lambda^*}
\end{equation}
mit $x$ der Position entlang des Wellenvektors, $\mu_0$ dem Brechungsindex im ungestörten Zustand, $\mu$ der maximalen Abweichung davon und $\lambda^*$ der Wellenlänge des Schalls. Das Isooktan bildet somit ein sinusförmiges Phasengitter, d.h. kohärentes Licht
\begin{equation}
 A e^{i \omega t}
\end{equation}
das senkrecht auf das Bad trifft, hat nach Durchquerung einen Phasenversatz
\begin{equation}
 A e^{i \omega \left( t - L \frac{\mu \left( x \right)}{c} \right)}
\end{equation}
der ebenfalls sinusförmig ist ($L$ ist hierbei die Breite des Bades). 


Raman-Nath-Theorie(Besselfunktionen)
