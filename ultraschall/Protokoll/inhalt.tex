\section{Aufgabenstellung}
\begin{enumerate}
 \item Bestimmung der Gitterkonstanten eines Sinusgitters aus dem Abstand der 1. Beugungsordnung.
\item Bestimmung der Gitterkonstanten von 5 Amplitudengittern.
\item Berechnung der Aperturfunktion für Gitter Nr.1 (größte Gitterkonstante,
      höchste Dichte an Beugungsmaxima) aus den ermittelten Intensitäten der
      Beugungsordnungen und Zeichnen einer Periode der Aperturfunktion.
\item Bestimmung des Verhältnisses der Spaltbreite zum Spaltabstand aus der
      Aperturfunktion.
\item Bestimmung des Auflösungsvermögens der Gitter bei ihrer vollen Ausleuchtung.

  \begin{enumerate}
    \item Messung der Intensitätsverteilung der Beugungsfigur eines Ultraschallwellengitters (Phasengitter)
	in Abhängigkeit von der Spannung am Ultraschallschwingquarz.
    \item Vergleich der Messergebnisse mit der Raman-Nath-Theorie.
    \item Bestimmung der Schallwellenlänge in Isooktan durch Ausmessen der Beugungsordnungen und Vergleich mit dem rechnerischen Wert.
  \end{enumerate}
\end{enumerate}

\section{Theorie}
