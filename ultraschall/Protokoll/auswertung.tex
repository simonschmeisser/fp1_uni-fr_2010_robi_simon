\section{Messwerte und Auswertung}

\subsection{Sinusgitter} %fertig

Der Laser wird anhand des 2. Spiegels zentriert, dann werden das Sinusgitter und der Schirm eingebaut. Die Distanz zwischen den beiden Maxima 1. Ordnung ergibt sich aus der $x$- und der $y$-Verschiebung auf dem Schirm. Wir haben folgende Werte gemessen:
\begin{itemize}
\item $D_x = (9.9 \pm 0,1) cm$ (Distanz in $x$-Richtung)
\item $D_y = (1.1 \pm 0.1) cm$ (Distanz in $y$-Richtung)
\item $L = (5.9 \pm 0.3) cm$ (Distanz zwischen dem Gitter und dem Schirm)
\end{itemize}

Aus $D_x$ und $D_y$ ergibt sich die Distanz $D=\sqrt{D_x^2 + D_y^2} = 9.96$ mit dem Fehler
$$\sigma_D = \sqrt{\left(\frac{\partial D}{\partial D_x}\right)^2\sigma_{D_x}^2 + \left(\frac{\partial D}{\partial D_y}\right)^2\sigma_{D_y}^2} = \sqrt 2 \left(\frac{\partial D}{\partial D_x}\right)\sigma_{D_x} = \sqrt 2 \frac{D_x}{\sqrt{D_x^2 + D_y^2}}\sigma_{D_x} = \sqrt 2 \frac{D_x}{D}\sigma_{D_x} = 0,14$$

Somit: $$\boxed{D = (9.96 \pm 0.14) cm}$$

Es folgt der Winkel $\alpha$ aus $\tan(\alpha) = \frac{D}{2L} \Leftrightarrow\alpha = \arctan\left(\frac{D}{2L}\right) = 40.17^\circ$ mit dem Fehler

$$\sigma(\alpha) = \sqrt{\left(\frac{\partial \alpha}{\partial D}\right)^2\sigma_{D}^2 + \left(\frac{\partial \alpha}{\partial L}\right)^2\sigma_{L}^2} = \sqrt{\frac{L\sigma_D^2 + D\sigma_L^2}{2L(1+D^2/4L^2)}} = 0.22^\circ$$

Es folgt: $$\boxed{\alpha = (40.17 \pm 0.22)^\circ = (0.701 \pm 0.004) rad} $$

Die Gitterkonstante $K$ l\"asst sich somit berechnen: $$ K\sin(\alpha) = n\lambda \Leftrightarrow K = \frac{n\lambda}{\sin(\alpha)} = 9810.13 \ \mathring A \ \text{mit \ } n=1$$

$K$ besitzt den Fehler (unter Vernachl\"assigung des Fehlers auf $\lambda$): $$\sigma_K = \sqrt{\left(\frac{\partial K}{\partial \alpha}\right)^2\sigma_\alpha^2} = \frac{\lambda\cos(\alpha)\sigma_\alpha}{\sin^2(\alpha)} = 44.62 \ \mathring A$$

Somit folgt: $$\boxed{K=(9810 \pm 44) \ \mathring A = (981.0 \pm 4.4) \ nm}$$

Als Vergleich: Auf dem Gitter ist die Zahl der Linien pro Millimeter angegeben: \\ $g_{theo}=1016 L/mm$. Es folgt der Wert f\"ur die Gitterkonstante: $K_{theo} = 1/g_{theo} = 9842.52 \ \mathring A$, welcher sehr dicht an unserem ermittelten Wert liegt und innerhalb der ersten Standardabweichung.

\subsection{Bestimmung der Gitterkonstanten}

Wir haben den Strahlengang wie in 3.2.2 beschrieben justiert um eine m\"oglichst hohe Genauigkeit bei unserer Messung zu erhalten.

\subsubsection{Eichung anhand des Gitters R}

Sei $\Delta t$ die zeitliche Distanz zwischen 2 Nebenmaxima der Intensit\"atsverteilung. Die Distanz zwischen dem Hauptmaximum und einem Nebenmaximum ist somit $\Delta t/2$ und ist proportional zu $\sin(\alpha)$. Es gilt: $$\sin(\alpha) = \frac{n\lambda}{K} = \gamma \frac{\Delta t}{2}$$
Wir bestimmen die Proportionalit\"atskonstante $\gamma$ durch lineare Regression, indem wir $\Delta t/2$ gegen $n$ auftragen und den Steigungsfaktor $b$ ermitteln. Der Achsenabschnitt $a$ sollte idealerweise Null sein.

$$\frac{\Delta t}{2} = \frac{\lambda}{K\gamma}n = b\cdot n$$

\begin{center}
\begin{tabular}{lllll}
\toprule
Peak & t in $\mu s$ & st in $\mu s$ & $\Delta t$ in $\mu s$ & $s\Delta t$ in $\mu s$ \\
\midrule
0. Ordnung \\
 & 0,0005010635 & 3,36E-008\\
\midrule
1. Ordnung\\
links & 0,0004047489 & 3,71E-008 & 0,0001924441 & 7,29084668847E-008\\
rechts & 0,000597193 & 3,59E-008\\
\midrule
2. Ordnung\\ 
links & 0,0003093735 & 2,07E-007 & 0,000383268 & 3,90324905236E-007\\
rechts & 0,0006926415 & 1,83E-007\\
\midrule
3. Ordnung\\ 
links & 0,0002143283 & 5,28E-007 & 0,0005732411 & 1,21769040489E-006\\
rechts & 0,0007875694 & 6,89E-007\\
\bottomrule
\end{tabular}
\end{center}

Die Parameter der linearen Regression:
$$ b               = 0.000190698      \pm 1.965E-07    (0.103\%) $$
$$ a               = 1.74811E-06      \pm 2.087E-07    (11.94\%) $$

$$ \gamma = \frac{\lambda}{Kb} = $$

%Auswertung

\subsubsection{Bestimmung der Gitterkonstanten f\"ur die 5 Gitter}
\begin{itemize}
\item G1
\item G3
\item G4
\item PHYWE-08534
\item PHYWE-08540
\end{itemize}

F\"ur die beiden PHYWE-Gitter kennen wir den theoretischen Wert, n\"amlich: 
\begin{itemize}
\item PHYWE-08534: $g_{theo} = 8 L/mm\ \Rightarrow \ K_{theo} = 125 \ \mu m$
\item PHYWE-08540: $g_{theo} = 10 L/mm\ \Rightarrow \ K_{theo} = 100 \ \mu m$
%Vergleich mit dem exp. ermittelten Wert
\end{itemize} 

\subsection{Berechnung der Aperturfunktion f\"ur Gitter 1}

Wir bestimmen die Aperturfunktion aus den Intensit\"aten der gemessenen Verteilung, indem wir diese invers fouriertransformieren. Die Inverse Fouriertransformation lautet:
$$g(x) = \frac{\sqrt{I_0}}{2} + \sum_{n=1}^N \sqrt{I_n}\cos\left(\frac{2\pi n}{K}x \right)$$
F\"ur die $I_0$ haben wir folgende Werte ermittelt:
%Resultate I_0
Mit $K = ... $ %Wert | erhalten wir....... + Plot 

\subsection{Bestimmung des Verh\"altnisses zwischen Spaltbreite und Spaltabstand}

\subsection{Aufl\"osungsverm\"ogen der Gitter}

Das Aufl\"osungverm\"ogen $a$ ergibt sich bei normalen Gittern durch $a=N\cdot m$, wobei N die Zahl der durchleuchteten Maxima ist, und m die Zahl der sichtbaren Maxima. N bestimmen wir, indem wir die Laserbreite $b_L$ messen und durch die Gitterkonstante $K$ teilen.\\
Die Laserbreite haben wir mit dem Schirm gemessen und haben $b_L = (3.0 \pm 0.5) mm$ erhalten.\\
Es gilt also: $$N = \frac{b_L}{K}$$
mit dem Fehler: $$\sigma_N = \sqrt{\frac{\sigma_{b_L}^2K^2 + \sigma_K^2b_L^2}{K^4}}$$
Der Fehler auf $a$ ist: $$\sigma_a = \frac{\partial a}{\partial N}\sigma_N = m\cdot \sigma_N$$

Somit folgt f\"ur die Gitter:

\begin{itemize}
\item
\item
\item
\item
\item
\end{itemize}





























